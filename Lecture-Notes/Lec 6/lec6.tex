\documentclass[twoside]{article}

%
% This is a borrowed LaTeX template file for lecture notes for CS267,
% Applications of Parallel Computing, UCBerkeley EECS Department.
% Now being used for CMU's 10725 Fall 2012 Optimization course
% taught by Geoff Gordon and Ryan Tibshirani.  When preparing 
% LaTeX notes for this class, please use this template.
%
% To familiarize yourself with this template, the body contains
% some examples of its use.  Look them over.  Then you can
% run LaTeX on this file.  After you have LaTeXed this file then
% you can look over the result either by printing it out with
% dvips or using xdvi. "pdflatex template.tex" should also work.
%

\setlength{\oddsidemargin}{0.25 in}
\setlength{\evensidemargin}{-0.25 in}
\setlength{\topmargin}{-0.6 in}
\setlength{\textwidth}{6.5 in}
\setlength{\textheight}{8.5 in}
\setlength{\headsep}{0.75 in}
\setlength{\parindent}{0 in}
\setlength{\parskip}{0.1 in}

%
% ADD PACKAGES here:
%

\usepackage{amsmath,amsfonts,graphicx}

%
% The following commands set up the lecnum (lecture number)
% counter and make various numbering schemes work relative
% to the lecture number.
%
\newcounter{lecnum}
\renewcommand{\thepage}{\thelecnum-\arabic{page}}
\renewcommand{\thesection}{\thelecnum.\arabic{section}}
\renewcommand{\theequation}{\thelecnum.\arabic{equation}}
\renewcommand{\thefigure}{\thelecnum.\arabic{figure}}
\renewcommand{\thetable}{\thelecnum.\arabic{table}}

%
% The following macro is used to generate the header.
%
\newcommand{\lecture}[4]{
   \pagestyle{myheadings}
   \thispagestyle{plain}
   \newpage
   \setcounter{lecnum}{#1}
   \setcounter{page}{1}
   \noindent
   \begin{center}
   \framebox{
      \vbox{\vspace{2mm}
    \hbox to 6.28in { {\bf Econ 626: Quantitative Methods II
  \hfill Fall 2018} }
       \vspace{4mm}
       \hbox to 6.28in { {\Large \hfill Lecture #1: #2  \hfill} }
       \vspace{2mm}
       \hbox to 6.28in { {\it Lecturer: #3 \hfill Scribes: #4} }
      \vspace{2mm}}
   }
   \end{center}
   \markboth{Lecture #1: #2}{Lecture #1: #2}

   %{\bf Note}: {\it LaTeX template courtesy of UC Berkeley EECS dept.}

   {\bf Disclaimer}: {\it Zhikun is fully responsible for the errors and typos appeared in the notes.}
   \vspace*{4mm}
}
%
% Convention for citations is authors' initials followed by the year.
% For example, to cite a paper by Leighton and Maggs you would type
% \cite{LM89}, and to cite a paper by Strassen you would type \cite{S69}.
% (To avoid bibliography problems, for now we redefine the \cite command.)
% Also commands that create a suitable format for the reference list.
\renewcommand{\cite}[1]{[#1]}
\def\beginrefs{\begin{list}%
        {[\arabic{equation}]}{\usecounter{equation}
         \setlength{\leftmargin}{2.0truecm}\setlength{\labelsep}{0.4truecm}%
         \setlength{\labelwidth}{1.6truecm}}}
\def\endrefs{\end{list}}
\def\bibentry#1{\item[\hbox{[#1]}]}

%Use this command for a figure; it puts a figure in wherever you want it.
%usage: \fig{NUMBER}{SPACE-IN-INCHES}{CAPTION}
\newcommand{\fig}[3]{
      \vspace{#2}
      \begin{center}
      Figure \thelecnum.#1:~#3
      \end{center}
  }
% Use these for theorems, lemmas, proofs, etc.
\newtheorem{theorem}{Theorem}[lecnum]
\newtheorem{lemma}[theorem]{Lemma}
\newtheorem{proposition}[theorem]{Proposition}
\newtheorem{claim}[theorem]{Claim}
\newtheorem{corollary}[theorem]{Corollary}
\newtheorem{definition}[theorem]{Definition}
%\newtheorem{example}[theorem]{Example}
\newenvironment{proof}{{\bf Proof:}}{\hfill\rule{2mm}{2mm}}
\newenvironment{example}{{\bf Example:}}{\hfill\rule{2mm}{2mm}}

\newtheorem{remark}[theorem]{Remark}
%\newenvironment{remark}[1][Remark]{\begin{trivlist}\item[\hskip \labelsep {\bfseries #1}]}{\end{trivlist}}


% **** IF YOU WANT TO DEFINE ADDITIONAL MACROS FOR YOURSELF, PUT THEM HERE:

\newcommand\E{\mathbb{E}}
\newcommand\dd{\mathrm{d}}

\usepackage{hyperref}
\usepackage{cancel}
\newcommand\pp{\partial}
\newcommand\pd{\partial}
\newcommand\imp{$\Longrightarrow$}
\newcommand\lb{\left (}
\newcommand\rb{\right )}
\newcommand\lsb{\left [}
\newcommand\rsb{\right ]}
\newcommand\lcb{\left \{}
\newcommand\rcb{\right \}}


\begin{document}
%FILL IN THE RIGHT INFO.
%\lecture{**LECTURE-NUMBER**}{**DATE**}{**LECTURER**}{**SCRIBE**}
\lecture{6}{More about Difference Equations}{Prof. Daniel Levy}{Zhikun Lu}
%\footnotetext{These notes are partially based on those of Nigel Mansell.}
\footnotetext[1]{Visit \url{http://www.luzk.net/misc} for updates.}

\hfill Date: September 5, 2018

\section{First Order Linear Difference Equations}
\begin{equation}
    \underbrace{y_{t}}_{Endog.} = \lambda y_{t-1} + b \underbrace{x_t}_{Exog.} + a, \qquad a, b, \lambda = \text{ parameters, } \lambda \neq 1
\end{equation}
\underline{Note}:

$\lambda = 1$ unit root, non-stationary, random walk

\underline{Sargent 1979, CH.9}

\begin{equation}
    y_{y} - \lambda y_{t-1} = b x_{t} + a
\end{equation}
\begin{equation}
    (1- \lambda L)y_{t} = b x_{t} + a
\end{equation}
\begin{equation}
    y_{t} = \frac{1}{1 - \lambda L}(b x_{t}+a) + {c \lambda^{t}}
\end{equation}
where ${c \lambda^{t}}$ is the ``summation constant'', similar to the constant in indefinite integral. We can check that $$(1- \lambda L){c \lambda^{t}} = c \lambda^{t} - \lambda L (c \lambda^{t}) = c \lambda^{t} - \lambda c \lambda^{t-1} = 0$$

\underline{Note}
\begin{eqnarray}
    L(x_{t} \pm y_{t}) = x_{t-1} \pm y_{t-1}\\
    L(a x_{t}) = a L x_{t} =  a x_{t-1}\\
    y_t = (\frac{1}{1- \lambda L})b x_{t} + (\frac{1}{1- \lambda L}) a + c \lambda^t
\end{eqnarray}

\underline{Assuptions}

$|\lambda| < 1 \Longrightarrow La = a$.

Then
\begin{equation}
   ( \frac{1}{1- \lambda L})a = (\frac{1}{1- \lambda})a
\end{equation}
This is because
\begin{equation}
    ( \frac{1}{1- \lambda L})a = (\sum_{i=0}^{\infty} \lambda^{i}L^{i})a = \sum_{i=0}^{\infty} (\lambda^{i}L^{i}a) = \sum_{i=0}^{\infty} (\lambda^{i}a) = a \sum_{i=0}^{\infty} \lambda^{i} = a(\frac{1}{1- \lambda})
\end{equation}
Thus, 
\begin{equation}
\begin{aligned}
    y_t &= (\frac{1}{1- \lambda L})b x_{t} + (\frac{1}{1- \lambda L}) a + c \lambda^t\\
        &= b \sum_{i=0}^{\infty} \lambda^i x_{t-i} + \frac{a}{1- \lambda} + c \lambda^t
\end{aligned}
\end{equation}

\underline{Recall}

If $|\lambda|<1$, then $(\frac{1}{1- \lambda L}) x_{t} = \sum_{i=0}^{\infty} \lambda^i x_{t-i}$

If $|\lambda|>1$, then $(\frac{1}{1- \lambda L}) x_{t} = -\sum_{i=1}^{\infty} {(\frac{1}{\lambda})}^i x_{t+i}$

\underline{Note}: If $|\lambda| > 1$, then we would use forward-looking solution.

\begin{example} Cagan (1956)
\begin{equation}
    \begin{cases}
        m_t &= \ln M_{t}\\
        p_{t} &= \ln P_{t}\\
        p^e_{t+1} &= \ln P^{e}_{t+1}  
    \end{cases}
\end{equation}
\underline{Money Market}: Money demand
\begin{equation}
    \underbrace{m_t - p_t}_{\ln (\frac{M_t}{P_t})} = \alpha (\underbrace{p^e_{t+1} - p_{t}}_{\text{expeced inflation}}), \qquad \alpha <0
\end{equation}
this is a(n) simplification/approximation under hyperinflation:
\begin{equation}
    (\frac{M}{P})^{d} = \frac{M}{P}(i, y) = \frac{M}{P}(r + {\pi^{e}}, y) 
\end{equation}
When inflation is high, $r$ and $y$ can be treated as fixed.

\underline{Assumption}
$$y = \bar{y}, r = \bar{r}$$
\begin{equation}
    P^{e}_{t+1} = (1-\gamma) P_{t} + \gamma P_{t-1} \Longrightarrow \pi_{t+1}^{e} = \Gamma \pi_{t}, \quad \Gamma < 0
\end{equation}
which is an example of \underline{extrapolative} expectation.
\begin{equation}
    p_{t+1}^{e} = (1+ \Gamma)p_{t} - \Gamma p_{t-1}
\end{equation}
\begin{equation}
    m_{t} - p_{t} = \alpha \Gamma (p_{t}-p_{t-1})
\end{equation}
\begin{equation}
    \Longrightarrow p_{t} - \underbrace{(\frac{a \Gamma}{1 + a \Gamma})}_{\lambda} p_{t-1} = \frac{1}{1 + a \Gamma} m_{t}
\end{equation}
\begin{equation}
    p_{t} - \lambda p_{t-1} = \frac{1}{1 + a \Gamma} m_{t}
\end{equation}
\begin{equation}
    p_{t} = \frac{1}{1 + a \Gamma} (\frac{1}{1 - \lambda L})m_{t} + c \lambda^{t}
\end{equation}
Since $|\lambda| = |\frac{\alpha \Gamma}{1+ \alpha \Gamma} | < 1$, \imp
\begin{equation}
    p_{t} = \frac{1}{1 + a \Gamma} \sum_{i=0}^{\infty} \lambda^{i} m_{t-i} + c \lambda^{t} = \frac{1}{1 + a \Gamma} \sum_{i=0}^{\infty} (\frac{\alpha \Gamma}{1+ \alpha \Gamma})^{i} m_{t-i} + c (\frac{\alpha \Gamma}{1+ \alpha \Gamma})^{t}
\end{equation}
\end{example}

\begin{example}
\underline{Perfect forsight}
    \begin{equation}
        p^{e}_{t+1} - p_{t} = p_{t+1} - p_{t}
    \end{equation}
    \begin{equation}
        \pi^{e}_{t+1} = \pi_{t+1}
    \end{equation}
    Then money demand is 
    \begin{equation}
        m_{t} - p_{t} = \alpha (p_{t+1} - p_{t})
    \end{equation}
    \begin{equation}
        \Longrightarrow p_{t+1} - \frac{\alpha-1}{\alpha}p_{t} = \frac{m_{t}}{\alpha}
    \end{equation}
    \begin{equation}
        \Longrightarrow (1 - \lambda L) p_{t+1} = \frac{m_{t}}{\alpha}
    \end{equation}
    Here, $|\lambda| = | \frac{\alpha-1}{\alpha}| > 1$. Hence use backward-looking solution:
    \begin{equation}
        \begin{aligned}
            p_{t+1} 
            &= - \frac{1}{\alpha} \left ( \frac{1}{1 - \lambda L} \right ) m_{t} + c \lambda^{t}\\
            &= - \frac{1}{\alpha} \sum_{i=1}^{\infty} \left ( \frac{1}{\lambda} \right )^i m_{t+i} + c \lambda^{t}\\
            &= - \frac{1}{\alpha} \sum_{i=1}^{\infty} \left ( \frac{1}{(\frac{\alpha-1}{\alpha})} \right )^i m_{t+i} + c (\frac{\alpha-1}{\alpha})^{t}\\
            &= \frac{1- \alpha}{\alpha^2} \sum_{i=1}^{\infty} \left ( \frac{\alpha}{\alpha-1} \right )^{i+1} m_{t+i} + c (\frac{\alpha-1}{\alpha})^{t}\\
            p_{t} &= \frac{1- \alpha}{\alpha^2} \sum_{i=1}^{\infty} \left ( \frac{\alpha}{\alpha-1} \right )^{i+1} m_{t+i-1} + c (\frac{\alpha-1}{\alpha})^{t-1}
        \end{aligned}
    \end{equation}
\end{example}

\section{Dynamic Discrete Time Infinite Horizon Model (Ramsey)}
\begin{equation}
    \max \quad \sum_{t=1}^{\infty} \beta^{t} u(c_{t}), \qquad 1 > \beta > 0
\end{equation}
\begin{equation}
    \text{s.t.} \quad c_t + B_{t} = (1+ \rho_{t-1}) B_{t-1} + y_{t}
\end{equation}
\begin{equation}
    \mathcal{L} = \sum_{t=1}^{\infty} \beta^{t} u(c_{t}) -  \sum_{t=1}^{\infty} \lambda_{t} [(1+ \rho_{t-1}) B_{t-1} + y_{t} - c_t - B_{t}]
\end{equation}
\underline{Choice}: $\{c_{t}, B_{t}\}_{t=1}^{\infty}$

\underline{FONC}:
\begin{eqnarray}
    &[c_{t}]& \qquad \beta^{t} u'(c_{t}) - \lambda_{t} = 0\\
    &[B_t]& \qquad -\lambda_{t} + \lambda_{t+1}(1+\rho_{t}) = 0
\end{eqnarray}
\imp
\begin{equation}
    1 + \rho_{t} = \frac{u'(c_{t})}{\beta u'(c_{t+1})}
\end{equation}
which says ``objective rate of substitution'' = ``subjective rate of substitution''.

\underline{Assumption}: $\rho_{t} = \rho, ~ \forall t$ 
\begin{equation}
    (1 - (1+ \rho)L)B_{t} = y_{t} - c_{t}
\end{equation}
\begin{eqnarray}
    B_{t} 
    &=& \frac{1}{1 - (1+ \rho)L}(y_{t} - c_{t}) + d (1 + \rho)^{t} \\
    &=& - \sum_{s=1}^{\infty} (\frac{1}{1+\rho})^{s}(y_{t+s}-c_{t+s}) + d (1 + \rho)^{t}\\
    &=& - \sum_{s=1}^{\infty} (\frac{1}{1+\rho})^{s}(y_{t+s}-c_{t+s}) \qquad (d = 0 \text{ because of NPGC})\\
    \sum_{s=1}^{\infty} (\frac{1}{1+\rho})^{s} c_{t+s} &=& B_{t} + \sum_{s=1}^{\infty} (\frac{1}{1+\rho})^{s} y_{t+s} 
\end{eqnarray}
which is the life-time budget constaint.

\underline{NPGC}:
\begin{equation}
    \lim_{t \to \infty} d(1+\rho)^{t} = 0 \Longrightarrow d = 0
\end{equation}

\underline{Assumption} $y_{t} = y, ~u(c) = \ln(c)$ \imp
\begin{equation}
    1 + \rho = \frac{1}{\beta }\frac{c_{t+1}}{c_{t} } \iff (1 + \rho)\beta c_{t} = c_{t+1}
\end{equation}
\begin{eqnarray}
    c_{t+1} &=& (1 + \rho)\beta c_{t}\\
    c_{t+2} &=& ((1 + \rho)\beta)^2 c_{t}\\
    c_{t+3} &=& ((1 + \rho)\beta)^3 c_{t}\\
    &\vdots&\\
    c_{t+s} &=& ((1 + \rho)\beta)^s c_{t}
\end{eqnarray}
\imp
\begin{equation}
    B_{t} + \sum_{s=1}^{\infty} (\frac{1}{1+\rho})^{s}y = 
     \sum_{s=1}^{\infty} (\frac{1}{1+\rho})^{s} (1 + \rho)^{s}\beta^{s} c_{t}
\end{equation}
\begin{equation}
    B_{t} + y \sum_{s=1}^{\infty} (\frac{1}{1+\rho})^{s} = 
     c_{t} \sum_{s=1}^{\infty} \cancel{(\frac{1}{1+\rho})^{s} (1 + \rho)^{s}}\beta^{s} 
\end{equation}
\begin{equation}
    B_{t} + \frac{1}{\rho}y = \frac{\beta}{1-\beta} c_{t}
\end{equation}
%Assume $\beta = \rho$, then we have
\begin{equation}
    c_t = \frac{1- \beta}{\beta \rho}(y + \rho B_{t})
\end{equation}

\section{Expectations}
How to get information about expectation?
\begin{enumerate}
    \item Survey
    \item Observe behaviour
    \item Assume some mechanism
    \begin{enumerate}
        \item Static expectation $p^{e}_{t+1} = p_{t}$\\
        Example: The cobweb model
        \item Extrapolative expectations\\
        Example:
        \begin{equation}
            p^{e}_{t+1} = \Gamma p_{t} + (1- \Gamma)p_{t-1} = \Gamma (p_{t} - p_{t-1}) + p_{t-1}
        \end{equation}
        Example:
        \begin{equation}
        \begin{aligned}
            C_{t+1} &= \alpha + \beta y^{e}_{t+1}\\
            &= \alpha + \beta \Gamma y_{t} + \beta (1- \Gamma) y_{t-1} + \epsilon_{t}
        \end{aligned}
        \end{equation}
        \item Adaptive expectations
    \end{enumerate}
\end{enumerate}

\underline{Adaptive Expectations}
\begin{equation}
    p^{e}_{t+1} = p^{e}_{t} + \theta (\underbrace{p_{t} - p^{e}_{t}}_{\text{forecast error}}), \qquad 0 < \theta < 1
\end{equation}
\begin{equation}
    p^{e}_{t+1} = \theta \sum_{s=0}^{\infty}(1- \theta)^{s} p_{t-s} + c (1 - \theta)^{t}
\end{equation}

\underline{Rational Expectations}
\begin{itemize}
    \item [1)] $P^{e}_{t+1} = \E [P_{t+1} \mid \Omega_{t}]$
    \item [2)] $P^{e}_{t+1} - P_{t+1} = \epsilon_{t+1}$, where $\E (\epsilon_{t+1}) = 0$, $\text{cov}(\epsilon_{t}, \epsilon_{t \pm 1})$.
\end{itemize}
\underline{Perfect Forsight} is an extreme case of RE  where $\epsilon_{t+1} = 0 ~ \forall t$.

\section{Stochastic Difference Equation}
\begin{equation}
    y_{t} = a \underbrace{\E [y_{t+1} | I_{t}]}_{\E_{t} y_{t+1}} + c x_{t}
\end{equation}
\begin{example}
    \begin{equation}
        p_{t} = (\frac{\alpha}{1+ \alpha}) \E [p_{t+1}| I_{t}]+     (\frac{1}{1+ \alpha})m_{t}
    \end{equation}
\end{example}

\underline{Law of Iterated Expectations}
\begin{equation}
    \E \lsb \E \lsb X | I_{t+1} \rsb | I_t \rsb = \E \lsb X| I_{t} \rsb
\end{equation}

\begin{eqnarray}
    y_{t} &=& a \E_{t}y_{t+1} + c x_{t}\\
    y_{t+1} &=& a \E_{t+1}y_{t+2} + c x_{t+1}\\
    \E_{t} y_{t+1} &=& a \E_{t} \E_{t+1}y_{t+2} + c \E_{t} x_{t+1} = a \E_{t} y_{t+2} + c \E_{t} x_{t+1} \\
    \Longrightarrow 
    y_{t} &=& a^{2} \E_{t} y_{t+2} + a c \E_{t} x_{t+1} + c x_{t}\\
    y_{t+2} &=& a \E_{t+2}y_{t+3} + c x_{t+2}\\
    \Longrightarrow 
    \E_{t} y_{t+2} &=& a \E_{t}y_{t+3} + c \E_{t} x_{t+2}\\
    \Longrightarrow 
    y_{t} &=& a^{3} \E_{t} y_{t+3} + a^2 c \E_{t} x_{t+2} + a c \E_{t} x_{t+1} + c x_{t}\\
    &\vdots& \notag\\
    y_{t} &=& c \sum_{i=0}^{T} a^{i}\E_{t}x_{t+i} + a^{T+1} \E_{t} y_{t+T+1}
\end{eqnarray}
Let $T \to \infty$ and assume that \begin{equation}
    \lim_{T \to \infty} [a^{T+1} \E_{t}y_{t+T+1}] = 0
\end{equation}
i.e. we are ruling out  bubble solution.

Then we get the fundamental solution:
\begin{equation}
    y_{t} = c \sum_{i=0}^{\infty} a^{i} \E_{t} x_{t+i}
\end{equation}

\section{Stochastic Dynamic Discrete Time Infinite Horizon Model}
\begin{equation}
    \max \quad \E \lsb \sum_{t=1}^{\infty} \beta^{t} u(c_{t})  \rsb 
\end{equation}
\begin{equation}
    \text{s.t.} \quad c_t + B_{t} = (1+ \rho_{t-1}) B_{t-1} + y_{t}
\end{equation}
\underline{Random Lagrangian Method} (Kushner (1965))

\underline{FONC}
\begin{eqnarray}
    \E_{t} [ \beta^{t} u'(c_{t}) - \lambda_{t}] = 0\\
    \E_{t} [ - \lambda_{t} + \lambda_{t+1} (1+ \rho_{t})] = 0
\end{eqnarray}

\underline{Note}: At time t, variables dated $t$ and earlier are known and hence they are not R.V.'s (random variables), i.e.
\begin{equation}
    \E_{t} x_{t} = \E [x_{t} | I_{t} ] = x_{t}
\end{equation}
\imp
\begin{eqnarray}
    \beta^{t} u'(c_{t}) - \lambda_{t} = 0\\
    -\lambda_{t} + (1 + \rho_{t}) \E_{t} \lambda_{t+1} = 0
\end{eqnarray}
\imp    
\begin{equation}
    u'(c_{t}) = (1 + \rho_{t}) \beta \E_{t} u'(c_{t+1})
\end{equation}
\begin{equation}
    \E_{t} u'(c_{t+1}) = \frac{1}{(1 + \rho_{t}) \beta} u'(c_{t})
\end{equation}
\begin{equation}
    u'(c_{t+1}) = \frac{1}{(1 + \rho_{t}) \beta} u'(c_{t}) + \epsilon_{t+1}
\end{equation}
with $\E_{t} \epsilon_{t+1} = 0$. Hence it's almost an AR(1) process.

\underline{Claim}: If $\E_t (x_{t+1}) = x_{t}$, then $x_{t+1} = x_{t} + \epsilon_{t+1}$, $E_{t} \epsilon_{t+1} = 0$.

\underline{Comment}: This is also a regression equation/model that can be used for estimation, taken to data for test directly. If $u(c) = \ln c$, then \begin{equation}
    \frac{1}{c_{t+1}} = \frac{1}{(1 + \rho_{t}) \beta}  \frac{1}{c_{t}} + \epsilon_{t+1}
\end{equation}
which is a regression equation with time-varying coefficient (Kalman filter). (F. Mishikin 1986, NBER-U.C. Press)

\underline{Assumption}: $\rho_{t} = \rho$ and $\beta = \frac{1}{1+ \rho}$

Then 
\begin{equation}
    u'(c_{t+1}) = u'(c_{t}) + \epsilon_{t+1}
\end{equation}
MU of consumption is a random walk.

With quadratic utility function, $\text{MU} = \alpha - \beta c$, Hall (1978) showed that
\begin{equation}
    c_{t+1} = c_{t} + \epsilon_{t+1}.
\end{equation}



%$$##
\clearpage
\section*{References}
\beginrefs
%\bibentry{CW87}{\sc D.~Coppersmith} and {\sc S.~Winograd}, 
%``Matrix multiplication via arithmetic progressions,''
%{\it Proceedings of the 19th ACM Symposium on Theory of %Computing},
%1987, pp.~1--6.
\bibentry{Sargent 1979}{\sc Tom Sargent}, 
``Chapter 9''
{\it Macroeconomic Theory},
1979
\endrefs

% **** THIS ENDS THE EXAMPLES. DON'T DELETE THE FOLLOWING LINE:

\end{document}





