\documentclass[twoside]{article}

%
% This is a borrowed LaTeX template file for lecture notes for CS267,
% Applications of Parallel Computing, UCBerkeley EECS Department.
% Now being used for CMU's 10725 Fall 2012 Optimization course
% taught by Geoff Gordon and Ryan Tibshirani.  When preparing 
% LaTeX notes for this class, please use this template.
%
% To familiarize yourself with this template, the body contains
% some examples of its use.  Look them over.  Then you can
% run LaTeX on this file.  After you have LaTeXed this file then
% you can look over the result either by printing it out with
% dvips or using xdvi. "pdflatex template.tex" should also work.
%

\setlength{\oddsidemargin}{0.25 in}
\setlength{\evensidemargin}{-0.25 in}
\setlength{\topmargin}{-0.6 in}
\setlength{\textwidth}{6.5 in}
\setlength{\textheight}{8.5 in}
\setlength{\headsep}{0.75 in}
\setlength{\parindent}{0 in}
\setlength{\parskip}{0.1 in}

%
% ADD PACKAGES here:
%

\usepackage{amsmath,amsfonts,graphicx}

%
% The following commands set up the lecnum (lecture number)
% counter and make various numbering schemes work relative
% to the lecture number.
%
\newcounter{lecnum}
\renewcommand{\thepage}{\thelecnum-\arabic{page}}
\renewcommand{\thesection}{\thelecnum.\arabic{section}}
\renewcommand{\theequation}{\thelecnum.\arabic{equation}}
\renewcommand{\thefigure}{\thelecnum.\arabic{figure}}
\renewcommand{\thetable}{\thelecnum.\arabic{table}}

%
% The following macro is used to generate the header.
%
\newcommand{\lecture}[4]{
   \pagestyle{myheadings}
   \thispagestyle{plain}
   \newpage
   \setcounter{lecnum}{#1}
   \setcounter{page}{1}
   \noindent
   \begin{center}
   \framebox{
      \vbox{\vspace{2mm}
    \hbox to 6.28in { {\bf Econ 626: Quantitative Methods II
  \hfill Fall 2018} }
       \vspace{4mm}
       \hbox to 6.28in { {\Large \hfill Lecture #1: #2  \hfill} }
       \vspace{2mm}
       \hbox to 6.28in { {\it Lecturer: #3 \hfill Scribes: #4} }
      \vspace{2mm}}
   }
   \end{center}
   \markboth{Lecture #1: #2}{Lecture #1: #2}

   %{\bf Note}: {\it LaTeX template courtesy of UC Berkeley EECS dept.}

   {\bf Disclaimer}: {\it Zhikun is fully responsible for the errors and typos appeared in the notes.}
   \vspace*{4mm}
}
%
% Convention for citations is authors' initials followed by the year.
% For example, to cite a paper by Leighton and Maggs you would type
% \cite{LM89}, and to cite a paper by Strassen you would type \cite{S69}.
% (To avoid bibliography problems, for now we redefine the \cite command.)
% Also commands that create a suitable format for the reference list.
\renewcommand{\cite}[1]{[#1]}
\def\beginrefs{\begin{list}%
        {[\arabic{equation}]}{\usecounter{equation}
         \setlength{\leftmargin}{2.0truecm}\setlength{\labelsep}{0.4truecm}%
         \setlength{\labelwidth}{1.6truecm}}}
\def\endrefs{\end{list}}
\def\bibentry#1{\item[\hbox{[#1]}]}

%Use this command for a figure; it puts a figure in wherever you want it.
%usage: \fig{NUMBER}{SPACE-IN-INCHES}{CAPTION}
\newcommand{\fig}[3]{
      \vspace{#2}
      \begin{center}
      Figure \thelecnum.#1:~#3
      \end{center}
  }
% Use these for theorems, lemmas, proofs, etc.
\newtheorem{theorem}{Theorem}[lecnum]
\newtheorem{lemma}[theorem]{Lemma}
\newtheorem{proposition}[theorem]{Proposition}
\newtheorem{claim}[theorem]{Claim}
\newtheorem{corollary}[theorem]{Corollary}
\newtheorem{definition}[theorem]{Definition}
%\newtheorem{example}[theorem]{Example}
\newenvironment{proof}{{\bf Proof:}}{\hfill\rule{2mm}{2mm}}
\newenvironment{example}{{\bf Example:}}{\hfill\rule{2mm}{2mm}}

\newenvironment{remark}{{\bf Remark:}}{\hfill\rule{2mm}{2mm}}
%\newtheorem{remark}[theorem]{Remark}

%\newenvironment{remark}[1][Remark]{\begin{trivlist}\item[\hskip \labelsep {\bfseries #1}]}{\end{trivlist}}


% **** IF YOU WANT TO DEFINE ADDITIONAL MACROS FOR YOURSELF, PUT THEM HERE:

\newcommand\E{\mathbb{E}}
\newcommand\dd{\mathrm{d}}

\usepackage{hyperref}
\usepackage{cancel}
\newcommand\pp{\partial}

\begin{document}
%FILL IN THE RIGHT INFO.
%\lecture{**LECTURE-NUMBER**}{**DATE**}{**LECTURER**}{**SCRIBE**}
\lecture{3}{Dynamic Programming III}{Prof. Daniel Levy}{Zhikun Lu}
%\footnotetext{These notes are partially based on those of Nigel Mansell.}
\footnotetext[1]{Visit \url{http://www.luzk.net/misc} for updates.}

\hfill Date: August 31, 2018

\section{Value function iteration}
\underline{Inital guess}: $V_0(K_{T+1}) = 0$
\begin{equation}
\Longrightarrow
    V_1(K_T) = \begin{cases}
        \max\limits_{\{C_T, K_{T+1}\}} [u(C_T)+ \beta V_0(K_{T+1})]\\
        \quad s.t. \quad C_T + K_{T+1} = K_T^\alpha
    \end{cases}
\end{equation}
Recall that $K_{T+1} = 0$ because it is the last period.
\begin{equation}
    \Longrightarrow \begin{cases}
        K_{T+1} = 0\\
        C_T =  K_T^\alpha
    \end{cases}
\end{equation}
Plugging (3.2) into (3.1),  we get
\begin{equation}
    V_1(K_T) = \ln (K_T^\alpha).
\end{equation}
Let's continue:
\begin{eqnarray}
    V_2(K_{T-1}) &=& \begin{cases}
        \max\limits_{\{C_{T-1}, K_{T}\}} [u(C_{T-1})+ \beta V_1(K_{T})]\\
        \quad s.t. \quad C_{T-1} + K_{T} = K_{T-1}^\alpha
    \end{cases}
    \\
    \Longrightarrow
    V_2(K_{T-1}) &=& \begin{cases}
        \max\limits_{\{C_{T-1}, K_{T}\}} u(C_{T-1})+ \beta \ln (K_T^\alpha)\\
        \quad s.t. \quad C_{T-1} + K_{T} = K_{T-1}^\alpha
    \end{cases}
\end{eqnarray}
\begin{equation}
    \mathcal{L} = \ln C_{T-1} + \beta \ln (K_T^\alpha) + \lambda [K_{T-1}^\alpha - C_{T-1} - K_{T}]
\end{equation}
\underline{FONC}
\begin{eqnarray}
    \frac{1}{C_{T-1}} - \lambda = 0\\
    \beta (\frac{1}{K_t^\alpha})(\alpha K_T^{\alpha-1}) - \lambda = 0   
\end{eqnarray}
\begin{eqnarray}
    \Longrightarrow
    \lambda &=& \frac{\alpha \beta}{K_T}\\
    \lambda &=& \frac{1}{C_{T-1}}\\
    \frac{\alpha \beta}{K_T} &=& \frac{1}{C_{T-1}} \quad \text{or} \quad C_{T-1} = \frac{K_T}{\alpha \beta}
\end{eqnarray}
\underline{Plug it into the constraint}
\begin{eqnarray}
    \frac{K_T}{\alpha \beta} + K_T = K_{T-1}^\alpha \\
    \Longrightarrow K_T = \frac{\alpha \beta}{1+\alpha \beta} K_{T-1}^\alpha 
\end{eqnarray}
\begin{equation}
    C_{T-1} = \frac{K_T}{\alpha \beta} = \frac{1}{\cancel{\alpha \beta}} \frac{\cancel{\alpha \beta}}{1+\alpha \beta} K_{T-1}^\alpha = \frac{1}{1+\alpha \beta} K_{T-1}^\alpha 
\end{equation}
Plug (3.12) and (3.14) into (3.5)
\begin{eqnarray}
    V_2(K_{T-1}) 
    &=& \max\limits_{\{C_{T-1}, K_{T}\}} u(C_{T-1})+ \beta \ln (K_T^\alpha) \quad \text{s.t.}~...\\
    &=& \ln( \frac{1}{1+\alpha \beta} K_{T-1}^\alpha ) + \beta \ln[\frac{\alpha \beta}{1+\alpha \beta} K_{T-1}^\alpha ]^\alpha\\
    &=& \alpha \beta \ln \alpha \beta - (1+ \alpha \beta) \ln (1+ \alpha \beta) + (1+ \alpha \beta) \ln K_{T-1}^\alpha
\end{eqnarray}
\begin{eqnarray}
    V_3(K_{T-2}) &=& \begin{cases}
        \max\limits_{\{C_{T-2}, K_{T-1}\}} [u(C_{T-2})+ \beta V_2(K_{T-1})]\\
        \quad s.t. \quad C_{T-2} + K_{T-1} = K_{T-2}^\alpha
    \end{cases}\\
    &\vdots&
\end{eqnarray}
It turns out that this sequence of value functions converges to:
\begin{equation}
    V(K_t) = \frac{\beta}{1- \beta}[\ln (1-\alpha \beta) + \frac{\alpha \beta}{1- \alpha \beta}\ln \alpha \beta] + \frac{\alpha}{1- \alpha \beta} \ln K_t
\end{equation}
To check if this limit function is indeed a solution, we plug it into the the Bellman equation (of the infinite horizon model):
\begin{equation}
    V(K_t) = \max \bigg \{ \ln C_t + \frac{\beta}{1- \beta}[\ln (1-\alpha \beta) + \frac{\alpha \beta}{1- \alpha \beta}\ln \alpha \beta] + \frac{\alpha \beta}{1- \alpha \beta} \ln K_{t+1} \bigg \} \quad s.t. \quad C_{t} + K_{t+1} = K_{t}^\alpha
\end{equation}
Recall
\begin{equation}
    \frac{1}{C_t} = \beta V'(K_{t+1}) \Longrightarrow \frac{1}{ \beta C_t } = V'(K_{t+1}) =  \frac{\alpha}{1- \alpha \beta} \frac{1}{K_{t+1}}~~\text{(by taking the derivative of (3.20))}
\end{equation}
Hence
\begin{equation}
    \frac{C_t}{K_{t+1}} = \frac{1- \alpha \beta}{\alpha \beta}
\end{equation}
Using the constaint $C_t + K_{t+1} = K_t^\alpha$,
we can get
\begin{equation}
    \frac{K_t^\alpha - K_{t+1}}{K_{t+1}} = \frac{1- \alpha \beta}{\alpha \beta} 
\end{equation}
\begin{equation}
    \Longrightarrow \text{saving rate} = \frac{K_{t+1}}{K_t^\alpha} = \alpha \beta
\end{equation}
same as old.

















































































%$$##
\clearpage
\section*{References}
%\beginrefs
%\bibentry{CW87}{\sc D.~Coppersmith} and {\sc S.~Winograd}, 
%``Matrix multiplication via arithmetic progressions,''
%{\it Proceedings of the 19th ACM Symposium on Theory of %Computing},
%1987, pp.~1--6.
%\endrefs

% **** THIS ENDS THE EXAMPLES. DON'T DELETE THE FOLLOWING LINE:

\end{document}








