% See instructions in the preamble file
\documentclass[twoside]{article}

%
% This is a borrowed LaTeX template file for lecture notes for CS267,
% Applications of Parallel Computing, UCBerkeley EECS Department.
% Now being used for CMU's 10725 Fall 2012 Optimization course
% taught by Geoff Gordon and Ryan Tibshirani.  When preparing 
% LaTeX notes for this class, please use this template.
%
% To familiarize yourself with this template, the body contains
% some examples of its use.  Look them over.  Then you can
% run LaTeX on this file.  After you have LaTeXed this file then
% you can look over the result either by printing it out with
% dvips or using xdvi. "pdflatex template.tex" should also work.
%

\setlength{\oddsidemargin}{0.25 in}
\setlength{\evensidemargin}{-0.25 in}
\setlength{\topmargin}{-0.6 in}
\setlength{\textwidth}{6.5 in}
\setlength{\textheight}{8.5 in}
\setlength{\headsep}{0.75 in}
\setlength{\parindent}{0 in}
\setlength{\parskip}{0.1 in}

%
% ADD PACKAGES here:
%

\usepackage{amsmath,amsfonts,graphicx}

%
% The following commands set up the lecnum (lecture number)
% counter and make various numbering schemes work relative
% to the lecture number.
%
\newcounter{lecnum}
\renewcommand{\thepage}{\thelecnum-\arabic{page}}
\renewcommand{\thesection}{\thelecnum.\arabic{section}}
\renewcommand{\theequation}{\thelecnum.\arabic{equation}}
\renewcommand{\thefigure}{\thelecnum.\arabic{figure}}
\renewcommand{\thetable}{\thelecnum.\arabic{table}}

%
% The following macro is used to generate the header.
%
\newcommand{\lecture}[4]{
   \pagestyle{myheadings}
   \thispagestyle{plain}
   \newpage
   \setcounter{lecnum}{#1}
   \setcounter{page}{1}
   \noindent
   \begin{center}
   \framebox{
      \vbox{\vspace{2mm}
    \hbox to 6.28in { {\bf Econ 626: Quantitative Methods II
  \hfill Fall 2018} }
       \vspace{4mm}
       \hbox to 6.28in { {\Large \hfill Lecture #1: #2  \hfill} }
       \vspace{2mm}
       \hbox to 6.28in { {\it Lecturer: #3 \hfill Scribes: #4} }
      \vspace{2mm}}
   }
   \end{center}
   \markboth{Lecture #1: #2}{Lecture #1: #2}

   %{\bf Note}: {\it LaTeX template courtesy of UC Berkeley EECS dept.}

   {\bf Disclaimer}: {\it Zhikun is fully responsible for the errors and typos appeared in the notes.}
   \vspace*{4mm}
}
%
% Convention for citations is authors' initials followed by the year.
% For example, to cite a paper by Leighton and Maggs you would type
% \cite{LM89}, and to cite a paper by Strassen you would type \cite{S69}.
% (To avoid bibliography problems, for now we redefine the \cite command.)
% Also commands that create a suitable format for the reference list.
\renewcommand{\cite}[1]{[#1]}
\def\beginrefs{\begin{list}%
        {[\arabic{equation}]}{\usecounter{equation}
         \setlength{\leftmargin}{2.0truecm}\setlength{\labelsep}{0.4truecm}%
         \setlength{\labelwidth}{1.6truecm}}}
\def\endrefs{\end{list}}
\def\bibentry#1{\item[\hbox{[#1]}]}

%Use this command for a figure; it puts a figure in wherever you want it.
%usage: \fig{NUMBER}{SPACE-IN-INCHES}{CAPTION}
\newcommand{\fig}[3]{
      \vspace{#2}
      \begin{center}
      Figure \thelecnum.#1:~#3
      \end{center}
  }
% Use these for theorems, lemmas, proofs, etc.
\newtheorem{theorem}{Theorem}[lecnum]
\newtheorem{lemma}[theorem]{Lemma}
\newtheorem{proposition}[theorem]{Proposition}
\newtheorem{claim}[theorem]{Claim}
\newtheorem{corollary}[theorem]{Corollary}
\newtheorem{definition}[theorem]{Definition}
%\newtheorem{example}[theorem]{Example}
\newenvironment{proof}{{\bf Proof:}}{\hfill\rule{2mm}{2mm}}
\newenvironment{example}{{\bf Example:}}{\hfill\rule{2mm}{2mm}}

\newtheorem{remark}[theorem]{Remark}
%\newenvironment{remark}[1][Remark]{\begin{trivlist}\item[\hskip \labelsep {\bfseries #1}]}{\end{trivlist}}


% **** IF YOU WANT TO DEFINE ADDITIONAL MACROS FOR YOURSELF, PUT THEM HERE:

\newcommand\E{\mathbb{E}}
\newcommand\dd{\mathrm{d}}

\usepackage{hyperref}
\usepackage{cancel}
\usepackage{listings}
\newcommand\pp{\partial}
\newcommand\pd{\partial}
\newcommand\imp{$\Longrightarrow$}
\newcommand\lb{\left (}
\newcommand\rb{\right )}
\newcommand\lsb{\left [}
\newcommand\rsb{\right ]}
\newcommand\lcb{\left \{}
\newcommand\rcb{\right \}}

\begin{document}

%\lecture{**LECTURE-NUMBER**}{**DATE**}{**LECTURER**}{**SCRIBE**}
\lecture{6}{Review Session \#6}{Aliaksandr Zaretski}{Zhikun Lu}
\footnotetext[1]{Visit \url{http://www.luzk.net/misc} for updates.}

\hfill Date: Date: September 5, 2018

%\tableofcontents
\begin{theorem}
    Consider \begin{equation}
        x^{(n)} + a_{n-1} x^{(n-1)} + ... + a_1 x' + a_0 = 0 \qquad [H]
    \end{equation}
    0 is a stable equilibrium point of [H] $\Longrightarrow a_0, a_1, ..., a_{n-1} > 0$.
\end{theorem}
\begin{theorem}[Routh--Hurwitz]
    See the scanned page.
\end{theorem}

\section{System of linear ODE}
\begin{equation}
    \underbrace{\dot{x}(t)}_{n \times 1} = \underbrace{A(t)}_{n \times n} \underbrace{x(t)}_{n \times 1} + \underbrace{B(t)}_{n \times m} \underbrace{u(t)}_{m \times 1}
\end{equation}
\begin{remark}
    Consider 
    $$y^{(n)} + a_{n-1}(t)y^{(n-1)}+...+a_{1}(t)y' + a_{0}(t)y = g(t) $$
    Let 
    \[
    \begin{cases}
        x_1 &\equiv y\\
        x_2 &\equiv y'\\
        &\vdots\\
        x_n &\equiv y^{(n-1)}
    \end{cases}
    \Longrightarrow
    \begin{cases}
        \dot{x}_1 &\equiv x_2\\
        \dot{x}_2 &\equiv x_3\\
        &\vdots\\
        \dot{x}_{n-1} &\equiv x_n
    \end{cases}
    \]

    \[  \Longrightarrow
        \dot{x}_n + a_{n-1}(t)x_n+...+a_{1}(t)x_2 + a_{0}(t)x_1 = g(t)
    \]

    \[
    x(t) = \begin{bmatrix}
       x_1(t) \\[0.3em]
       \vdots \\[0.3em]
       x_n(t)      
     \end{bmatrix}
     \qquad
    u(t) = \begin{bmatrix}
       u_1(t) \\[0.3em]
       \vdots \\[0.3em]
       u_m(t)      
     \end{bmatrix}
    \]

    \[
    A(t) = \begin{bmatrix}
       a_{11}(t)& \cdots & a_{1n}(t)    \\[0.3em]
       \vdots   &        & \vdots       \\[0.3em]
       a_{n1}(t)& \cdots & a_{nn}(t)    
     \end{bmatrix}
     \qquad
    B(t) = \begin{bmatrix}
       b_{11}(t)& \cdots & b_{1m}(t)    \\[0.3em]
       \vdots   &        & \vdots       \\[0.3em]
       b_{n1}(t)& \cdots & b_{nm}(t)    
     \end{bmatrix}
    \]
    Then we have
    \begin{equation}
    \underbrace{\begin{bmatrix}
            \dot{x}_1(t) \\[0.3em]
            \dot{x}_2(t) \\[0.3em]
            \vdots \\[0.3em]
            \dot{x}_{n-1}(t) \\[0.3em]
            \dot{x}_n(t)      
         \end{bmatrix}}_{\dot{x}(t)} 
    = \underbrace{\begin{bmatrix}
            0 & 1& 0& \cdots   & 0 & 0   \\[0.3em]
            0 & 0& 1& \cdots   & 0 & 0   \\[0.3em]
            \vdots & \vdots   & \vdots   &   \ddots     & \vdots& \vdots       \\[0.3em]
            0 & 0& 0& \cdots   & 0 & 1   \\[0.3em]
            -a_0(t) & -a_{1}(t)& -a_{2}(t)& \cdots & -a_{n-2}(t)& -a_{n-1}(t)    
          \end{bmatrix}}_{A(t)}
    \underbrace{\begin{bmatrix}
            x_1(t) \\[0.3em]
            x_2(t) \\[0.3em]
            \vdots \\[0.3em]
            x_{n-1}(t) \\[0.3em]
            x_n(t)      
         \end{bmatrix}}_{x(t)} + 
    \underbrace{\begin{bmatrix}
            0 \\[0.3em]
            0 \\[0.3em]
            \vdots \\[0.3em]
            0 \\[0.3em]
            1         
        \end{bmatrix} }_{B(t)} \underbrace{g(t)}_{u(t)}
    \end{equation}
\end{remark}

\begin{theorem}
    If all elements of $A(t), B(t), u(t)$ are continuous, then $\forall$ initial condition $x(t_{0}) = x^{0}$, $\exists$ a unique solution to [C]
\end{theorem}

\begin{remark}  
    \begin{equation}
        \dot{x}(t) = A(t)x(t)+ B(t)u(t) \qquad [C]
    \end{equation}
    \begin{equation}
        \dot{x}(t) = A(t)x(t)  \qquad [H]
    \end{equation}
\end{remark}

\begin{theorem}
    $\lcb H \rcb$ is a vector space.
\end{theorem}
\begin{proof}
    Excercise.
\end{proof}

\begin{theorem}
    $\dim \lcb H \rcb = n$.
\end{theorem}

\begin{definition}
    Let $\lcb x^1(t), ..., x^n(t) \rcb$ be a basis of $\lcb H \rcb$. Then \begin{equation}
        \Phi(t) = (x^1(t), ..., x^n(t)) = \begin{bmatrix}
            x_{1}^{1}(t)& \cdots & x_{1}^{n}(t)    \\[0.3em]
       \vdots   &        & \vdots       \\[0.3em]
       x_{n}^{1}(t)& \cdots & x_{n}^{n}(t)  
        \end{bmatrix}
    \end{equation}
    where $\Phi(t)$ is a fundamental matrix of [H].
\end{definition}

\section{Systems of linear ODE}
\begin{remark}
    \begin{enumerate}
        \item There are infinitely many fundamental matrices of [H]
        \item $\dot{\Phi}(t) \equiv (\dot{x}^1(t), ..., \dot{x}^n(t)) 
        = \begin{bmatrix}
            \dot{x}_{1}^{1}(t)& \cdots & \dot{x}_{1}^{n}(t)    \\[0.3em]
            \vdots   &        & \vdots       \\[0.3em]
            \dot{x}_{n}^{1}(t)& \cdots & \dot{x}_{n}^{n}(t)  
        \end{bmatrix}$
    \end{enumerate}
\end{remark}

\begin{lemma}
    $\Phi(t)$ is a fundamental matrix of [H] $\iff$ $\dot{\Phi}(t) = A(t)\Phi(t)$ and $\Phi(t)$ is invertible for all $t$.
\end{lemma}
\begin{proof}
    ``\imp''
    \begin{equation}
        \dot{\Phi}(t) = (\dot{x}^1(t), ..., \dot{x}^n(t)) = (A(t){x}^1(t), ..., A(t){x}^n(t)) = A(t) (x^1(t), ..., x^n(t)) = A(t) \Phi(t)
    \end{equation}
    ``$\Longleftarrow$''

    $\Phi(t)$ has linearly independent columns. Suppose $x(t)$ is a solution to [H] that satisfies $x(t_0) = x^{0}$. Consider $\tilde{x}(t) = \Phi(t)c$. Then $\tilde{x}(t_0) = \Phi(t_0) c \Longrightarrow c = \Phi(t_0)^{-1}\tilde{x}(t_0) $. 

    Suppose $\tilde{x}(t_0) = x^0$. Then $\tilde{x}(t) = \Phi(t) \Phi(t_0)^{-1} x^0 $. Then
    \begin{equation}
        \dot{\tilde{x}}(t) = \dot{\Phi}(t)\Phi(t_0)^{-1} x^0 = A(t) \underbrace{\Phi(t) \Phi(t_0)^{-1} x^0}_{\tilde{x}(t)} = A(t) \tilde{x}(t)
    \end{equation}
    Hence $\tilde{x}(t)$ is a solution to [H]. By uniqueness, \begin{equation}
        \tilde{x}(t) = x(t) = \Phi(t) \underbrace{\Phi(t_0)^{-1}x^0}_{c}
    \end{equation}
    Hence, $\Phi(t)$ has a basis of $\{ H \}$ as columns \imp fundamental matrix. 
\end{proof}


\begin{definition}
    $\Phi(t,t_{0})$ is a \underline{state transition matrix of } [H] if 
    \begin{enumerate}
        \item $\dot{\Phi}(t,t_0) = A(t) {\Phi}(t,t_0)$
        \item ${\Phi}(t_0,t_0) = I_{n}$
    \end{enumerate}
\end{definition}

\begin{theorem}
    If $\Phi(t)$ is a fundamental matrx of [H], then $\Phi(t,t_{0}) = \Phi(t)\Phi(t_0)^{-1}$ is a state transition matrx.
\end{theorem}
\begin{proof}
\begin{enumerate}
    \item $\dot{\Phi}(t,t_0) =  \dot{\Phi}(t)\Phi(t_0)^{-1} = A(t) {\Phi}(t) {\Phi}(t_0) = A(t) {\Phi}(t,t_0)$
    \item ${\Phi}(t_0,t_0) = {\Phi}(t_0){\Phi}(t_0)^{-1} = I_n$
\end{enumerate}
\end{proof}

\textbf{Remark}: State transition matrx is unique.
\begin{proof}
    Exercise.
\end{proof}

\begin{theorem}
    Suppose assumptions of theorem ???(7.1) holds. Then $x(t) = \Phi(t,t_0)x^0$ is the unique solution to [H] that satisfies $x(t_0) = x^{0}$.
\end{theorem}
\begin{proof}
    $x(t) = \Phi(t)c$ for some $c \in \mathbb{R}^n$ \imp $x(t_0) = \Phi(t_0)c$ \imp $c = \Phi(t_0)^{-1} x(t_{0})$

    \imp $x(t)= \Phi(t)\Phi(t_0)^{-1}x(t_0) = \Phi(t,t_0)x(t_0)$
\end{proof}

\begin{remark}
    $x(t) = \Phi(t) c, ~c \in \mathbb{R}^{n}$ is the general solution to [H].
\end{remark}

\begin{theorem}
    The general solution to [C] is 
    \begin{equation}
        x(t) = \Phi(t) c + \int_{t_{0}}^{t} \Phi(t,s) B(s) u(s) \dd s.
    \end{equation}
    Hence, the solution that satisfies $x_{t_{0}} = x^{0}$ is \begin{equation}
        x(t) = \Phi(t,t_{0}) x^{0} + \int_{t_{0}}^{t} \Phi(t,s) B(s) u(s) \dd s
    \end{equation}
\end{theorem}

\section{Systems of linear ODE with constant coefficients}
\begin{equation}
    \dot{x}(t) = A x(t) + B(t)u(t) \qquad [C]
\end{equation}
\begin{equation}
    \dot{x}(t) = A x(t) \qquad [H]
\end{equation}
\begin{definition}
    Let $A \in \mathbb{R}^{n\times n}$. Then $e^{A} \equiv \sum_{k=0}^{\infty} \frac{A^{k}}{k!}$ is the \underline{matrix exponential} of A.
\end{definition}
\begin{theorem}
    Let $A, B, C \in \mathbb{R}^{n\times n}$, and $c, d \in \mathbb{R}$.
    \begin{enumerate}
        \item Suppose $AB = BA$. Then \[
        e^{cA}e^{dB} = e^{cA+dB}
        \]
        In particular, $e^{cA}e^{dA} = e^{(c+d)A}$.
        \item $\det(e^{A}) = e^{tr{(A)}}$. Hence, $e^{A}$ is invertible $\forall A \in \mathbb{R}^{n \times n}$.
        \item Suppose $D$ is diagonal, i.e. \[
        D = diag(d_{i}) = \begin{bmatrix}
            d_{1} & & 0\\
            &\ddots& \\
            0 & & d_{n}
        \end{bmatrix}
        \]
        Then \[
        e^{D} = diag(e^{d_{i}}) = \begin{bmatrix}
            e^{d_{1}} & & 0\\
            &\ddots& \\
            0 & & e^{d_{n}}
        \end{bmatrix}.
        \]
        In particular, $e^{0_{n \times n}} = I_{n}, ~e^{I_{n}} = e I_{n}$.
        \item $(e^{A})^{-1} = e^{-A}$.
        \item Let $A = P J P^{-1}$, where $J$ -- Jordan normal form, then \[
        e^{A} = P e^{J} P^{-1}.
        \]
        Suppose $A$ is diagonalizable, $A = P D P^{-1}$, then $e^{A} = P e^{D} P^{-1}$
        \item  $
        \dfrac{\dd }{\dd t} \lb e^{At} \rb = A e^{At}.
        $
    \end{enumerate}
\end{theorem}
\begin{proof}
    \begin{itemize}
        \item 1. follows from Binomial formula / Cauchy product.
        \item 2. follows from Jacobi's formula.
        \item 3. -- 6. Exercise.
        \item For 4., use 1.
    \end{itemize}
\end{proof}












%=======================================































%=======================================

%$$##
%\clearpage
%\section*{References}
%\beginrefs
%\bibentry{CW87}{\sc D.~Coppersmith} and {\sc S.~Winograd}, 
%``Matrix multiplication via arithmetic progressions,''
%{\it Proceedings of the 19th ACM Symposium on Theory of %Computing},
%1987, pp.~1--6.
%\endrefs

% **** THIS ENDS THE EXAMPLES. DON'T DELETE THE FOLLOWING LINE:

\end{document}





