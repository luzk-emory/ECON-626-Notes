% See instructions in the preamble file
\documentclass[twoside]{article}

%
% This is a borrowed LaTeX template file for lecture notes for CS267,
% Applications of Parallel Computing, UCBerkeley EECS Department.
% Now being used for CMU's 10725 Fall 2012 Optimization course
% taught by Geoff Gordon and Ryan Tibshirani.  When preparing 
% LaTeX notes for this class, please use this template.
%
% To familiarize yourself with this template, the body contains
% some examples of its use.  Look them over.  Then you can
% run LaTeX on this file.  After you have LaTeXed this file then
% you can look over the result either by printing it out with
% dvips or using xdvi. "pdflatex template.tex" should also work.
%

\setlength{\oddsidemargin}{0.25 in}
\setlength{\evensidemargin}{-0.25 in}
\setlength{\topmargin}{-0.6 in}
\setlength{\textwidth}{6.5 in}
\setlength{\textheight}{8.5 in}
\setlength{\headsep}{0.75 in}
\setlength{\parindent}{0 in}
\setlength{\parskip}{0.1 in}

%
% ADD PACKAGES here:
%

\usepackage{amsmath,amsfonts,graphicx}

%
% The following commands set up the lecnum (lecture number)
% counter and make various numbering schemes work relative
% to the lecture number.
%
\newcounter{lecnum}
\renewcommand{\thepage}{\thelecnum-\arabic{page}}
\renewcommand{\thesection}{\thelecnum.\arabic{section}}
\renewcommand{\theequation}{\thelecnum.\arabic{equation}}
\renewcommand{\thefigure}{\thelecnum.\arabic{figure}}
\renewcommand{\thetable}{\thelecnum.\arabic{table}}

%
% The following macro is used to generate the header.
%
\newcommand{\lecture}[4]{
   \pagestyle{myheadings}
   \thispagestyle{plain}
   \newpage
   \setcounter{lecnum}{#1}
   \setcounter{page}{1}
   \noindent
   \begin{center}
   \framebox{
      \vbox{\vspace{2mm}
    \hbox to 6.28in { {\bf Econ 626: Quantitative Methods II
  \hfill Fall 2018} }
       \vspace{4mm}
       \hbox to 6.28in { {\Large \hfill Lecture #1: #2  \hfill} }
       \vspace{2mm}
       \hbox to 6.28in { {\it Lecturer: #3 \hfill Scribes: #4} }
      \vspace{2mm}}
   }
   \end{center}
   \markboth{Lecture #1: #2}{Lecture #1: #2}

   %{\bf Note}: {\it LaTeX template courtesy of UC Berkeley EECS dept.}

   {\bf Disclaimer}: {\it Zhikun is fully responsible for the errors and typos appeared in the notes.}
   \vspace*{4mm}
}
%
% Convention for citations is authors' initials followed by the year.
% For example, to cite a paper by Leighton and Maggs you would type
% \cite{LM89}, and to cite a paper by Strassen you would type \cite{S69}.
% (To avoid bibliography problems, for now we redefine the \cite command.)
% Also commands that create a suitable format for the reference list.
\renewcommand{\cite}[1]{[#1]}
\def\beginrefs{\begin{list}%
        {[\arabic{equation}]}{\usecounter{equation}
         \setlength{\leftmargin}{2.0truecm}\setlength{\labelsep}{0.4truecm}%
         \setlength{\labelwidth}{1.6truecm}}}
\def\endrefs{\end{list}}
\def\bibentry#1{\item[\hbox{[#1]}]}

%Use this command for a figure; it puts a figure in wherever you want it.
%usage: \fig{NUMBER}{SPACE-IN-INCHES}{CAPTION}
\newcommand{\fig}[3]{
      \vspace{#2}
      \begin{center}
      Figure \thelecnum.#1:~#3
      \end{center}
  }
% Use these for theorems, lemmas, proofs, etc.
\newtheorem{theorem}{Theorem}[lecnum]
\newtheorem{lemma}[theorem]{Lemma}
\newtheorem{proposition}[theorem]{Proposition}
\newtheorem{claim}[theorem]{Claim}
\newtheorem{corollary}[theorem]{Corollary}
\newtheorem{definition}[theorem]{Definition}
%\newtheorem{example}[theorem]{Example}
\newenvironment{proof}{{\bf Proof:}}{\hfill\rule{2mm}{2mm}}
\newenvironment{example}{{\bf Example:}}{\hfill\rule{2mm}{2mm}}

\newtheorem{remark}[theorem]{Remark}
%\newenvironment{remark}[1][Remark]{\begin{trivlist}\item[\hskip \labelsep {\bfseries #1}]}{\end{trivlist}}


% **** IF YOU WANT TO DEFINE ADDITIONAL MACROS FOR YOURSELF, PUT THEM HERE:

\newcommand\E{\mathbb{E}}
\newcommand\dd{\mathrm{d}}

\usepackage{hyperref}
\usepackage{cancel}
\usepackage{listings}
\newcommand\pp{\partial}
\newcommand\pd{\partial}
\newcommand\imp{$\Longrightarrow$}
\newcommand\lb{\left (}
\newcommand\rb{\right )}
\newcommand\lsb{\left [}
\newcommand\rsb{\right ]}
\newcommand\lcb{\left \{}
\newcommand\rcb{\right \}}

\begin{document}

%\lecture{**LECTURE-NUMBER**}{**DATE**}{**LECTURER**}{**SCRIBE**}
\lecture{8}{Review Session \#8}{Aliaksandr Zaretski}{Zhikun Lu}
\footnotetext[1]{Visit \url{http://www.luzk.net/misc} for updates.}

\hfill Date: Date: September 7, 2018

%\tableofcontents
\section{Linear DE}
\begin{equation}
    \varphi_{n}(L) = 1 + a_{1,n}L + ... + a_{k,n}L^{k}
\end{equation}
\[
\begin{cases}
    \varphi_{n}(L)x_{n} = g_{n}, &\text{ [C] }\\
    \varphi_{n}(L)x_{n} = 0    , &\text{ [H] }
\end{cases}
\iff
\begin{cases}
    x_n + a_{1,n}x_{n-1}+ ...+ a_{k,n} x_{n-k} = g_{n}, &\text{ [C] }\\
    x_n + a_{1,n}x_{n-1}+ ...+ a_{k,n} x_{n-k} = 0    , &\text{ [H] }    
\end{cases}
\]

\begin{theorem}
    Consider a k-th order linear DE
    \begin{enumerate}
        \item [(a)] $\{x_{n}\}$ is a general solution to [C] $\iff$ $x_{n} = x_{h,n}+ x_{p,n}$, where $\begin{cases}
            \{x_{h,n}\} &\text{-- general solution to [H] }\\
            \{x_{p,n}\} &\text{-- general solution to [C] }
        \end{cases}$
        \item [(b)] $\{ H \}$ is a vector space.
        \item [(c)] $\dim \{ H \} = k$.
        \item [(d)] Let $\{x_{n}^{1} \}, ..., \{x_{n}^{k} \}$ be solutions of [H] that satisfies $\begin{cases}
            x_{0}^{1} = 1, \\
            \\
            \\

        \end{cases}$

        Then $\{ \{x_{n}^{1} \}, ..., \{x_{n}^{k} \} \}$ is a basis of $\{H\}$.
    \end{enumerate}
\end{theorem}
\begin{proof}
    Follows from the arguments similar to Theorem 3.2-3.4 + COrollary exercise.
\end{proof}

\section{Linear DE with constant coefficients}
\begin{equation}
    \varphi_{n}(L) = 1 + a_{1}L + ... + a_{k}L^{k}
\end{equation}
\[
\begin{cases}
    \varphi_{n}(L)x_{n} = g_{n}, &\text{ [C] }\\
    \varphi_{n}(L)x_{n} = 0    , &\text{ [H] }
\end{cases}
\iff
\begin{cases}
    x_n + a_{1}x_{n-1}+ ...+ a_{k} x_{n-k} = g_{n}, &\text{ [C] }\\
    x_n + a_{1}x_{n-1}+ ...+ a_{k} x_{n-k} = 0    , &\text{ [H] }    
\end{cases}
\]

\begin{definition}
    $\lambda^{k} + a_{1} \lambda^{k-1} + ... + a_{k-1} \lambda + a_{k} = 0$ [CE] is the \underline{characteristic equation} corresponding to [H] (Assume $a^{k} \neq 0$).
\end{definition}

\begin{theorem}
    $\lambda$ is a solution to [CE] $\iff$ $\{ \lambda^{n} \}, \lambda \neq 0,$ is a solution to [H].
\end{theorem}
\begin{proof}
    ``\imp''

    $\lambda^{k} + a_{1} \lambda^{k-1} + ... + a_{k-1} \lambda + a_{k} = 0$ multiplied by $\lambda^{n-k}$ \imp 
    \[
    \lambda^{n} + a_{1} \lambda^{n-1} + ... + a_{k-1} \lambda^{n-k+1} + a_{k} \lambda^{n-k} = 0
    \]

    ``$\Longleftarrow$''

    $\lambda^{n} + a_{1} \lambda^{n-1} + ... + a_{k-1} \lambda^{n-k+1} + a_{k} \lambda^{n-k} = 0$ \imp $\lambda^{n-k} (...) = 0$ \imp True.
\end{proof}

\subsection*{Forming the basis of $\{H\}$}

Suppose $\lambda_{1}, ..., \lambda_{n}$ are roots of [CE]
\begin{enumerate}
    \item $\lambda_{j} \in \mathbb{R}$, distinct from other roots, then take $\{\lambda_{j}^{n}\}$
    \item $\underbrace{\lambda_{j}, ..., \lambda_{j+m-1}}_{m \text{ terms}} \in \mathbb{R}$, equal real roots, then take $\{\lambda_{j}^{n}\}$, $\{n\lambda_{j}^{n}\}$, ..., $\{n^{m-1}\lambda_{j}^{n}\}$.
    \item $\begin{cases}
        \\
        
    \end{cases}
    \begin{array}{lllllll}
        \lambda_{j}    &= a_{j} + b_{j} {i} & \Longrightarrow &  (a_{j}+b_{j} i)^{n} & = (\underbrace{|z|}_{\Gamma_{j}} e^{i \theta_{j}})^{n}   &= \Gamma_{j} e^{i n \theta_{j}} & = \Gamma^{n}_{j}(\cos(n \theta_{j}) + i \sin(n \theta_{j}))\\
        \lambda_{j+1}  &= a_{j} - b_{j} {i} & \Longrightarrow &  &  &  & \Rightarrow \Gamma^{n}_{j}(\cos(n \theta_{j}) - i \sin(n \theta_{j}))
    \end{array}$

    Then $c_{1} \lambda_{j}^{n} + c_{2} \lambda_{j+1}^{n}$ \imp $\Gamma^{n}_{j}(c_{j} \cos(n \theta_{j}) + c_{j+1}\sin(n \theta_{j}))$
    \item $\begin{cases}
        \lambda_{j}   = a_{j} + i b_{j}\\
        \lambda_{j+1} = a_{j} - i b_{j}\\
        \lambda_{j+2} = a_{j} + i b_{j}\\
        \lambda_{j+3} = a_{j} - i b_{j}        
    \end{cases}$ \imp $\Gamma^{n}_{j}(c_{j} \cos(n \theta_{j}) + c_{j+1}\sin(n \theta_{j}) + c_{j+2} n \cos(n \theta_{j}) + c_{j+3} n \sin(n \theta_{j}))$
\end{enumerate}

\underline{General solution to [H]}

$X_{h,n}$ is a linear combination of basis functions. For example,
\[
\begin{cases}
    x_n + a_{1}x_{n-1}+ ...+ a_{8} x_{n-8} + a_{9} x_{n-9} = 0      , &\text{ [C] }\\
    \lambda^{9} + a_{1}\lambda_{8}+ ...+ a_{8}  + \lambda a_{9} = 0    , &\text{ [H] }
\end{cases}
\]
\[
\begin{cases}
    \lambda_{1}, ..., \lambda_{9}  \text{ -- roots of [CE] }\\
    \lambda_{1} \neq \lambda_{2} \neq \lambda_{3} \in \mathbb{R}\\
    \lambda_{3} =    \lambda_{4} =    \lambda_{5} \in \mathbb{R}\\
    \lambda_{6} = a + b i\\
    \lambda_{7} = a - b i\\
    \lambda_{8} = a + b i\\
    \lambda_{9} = a - b i
\end{cases}
\]
\imp $$x_{h,n} = c_{1}\lambda_{1}^{n}+  c_{2}\lambda_{2}^{n}+ c_{3}\lambda_{3}^{n}+ c_{4} n \lambda_{4}^{n} + c_{5} n^{2} \lambda_{5}^{n} + \Gamma^{n}(c_{6}\cos(b \theta) + c_{7}\sin(b \theta)) + n \Gamma^{n}(c_{8}\cos(b \theta) + c_{9}\sin(b \theta))$$
where $\Gamma = \sqrt{a^{2}+b^{2}}.$
%=======================================



\clearpage
\underline{Particular solution to [C]}

\[
\phi(L)x_{n} = g_{n} \quad \text{[C]}
\]

\begin{table}[htbp]
\begin{tabular}{ll}
$g_n$ & Guess for $x_{p,n}$ \\ \hline
$C$ -- constant  & $D$ -- constant \\
$b^{n}$ & $D b^{n}$   \\
$\sin(At)$ & $D\sin(At)+E\cos(At)$ \\
$\cos(At)$ & $D\sin(At)+E\cos(At)$ \\
$n^d$ & $c_0+c_1 t + ... + c_d t^d$ \\
sum or product & sum or product \\
 of the above &  of the above  
\end{tabular}
\end{table}

\begin{remark}
If $x_{p,n}$ solves [H], then multiplt it by n.
\end{remark}











\begin{example}
    \[
    \begin{cases}
        x_{n} + 2x_{n-1} + 2 x_{n-2} = n^{2}, &\text{ [C] }\\
        \lambda^{2} + 2 \lambda + 2  = 0,     &\text{ [CE] }
    \end{cases}
    \]
    \imp $\begin{cases}
        \lambda_{1} = -1 + i\\    
        \lambda_{2} = -1 + i,       
    \end{cases} \qquad 
    \begin{cases}
        a = -1\\
        b = 1
    \end{cases}
    \Longrightarrow
    \Gamma = \sqrt{2}, ~ \theta = atan2(1,-1) = \dfrac{3\pi}{4}
    $

    \imp $x_{h,n} = (\sqrt{2})^{n} \left[ c_{1} \cos (\frac{3\pi n}{4}) + c_{2} \sin (\frac{3\pi n}{4}) \right]$

    Guess for $x_{p,n} = b_{0}+b_{1} n + b_{2} n^{2}$ \imp $\begin{cases}
        b_{0} = ...\\
        b_{1} = ...\\
        b_{2} = ...
    \end{cases}$

    ...
\end{example}

\begin{theorem}
    \begin{itemize}
        \item [(a)] If $1 + a_1 + ... + a_k \neq 0$, $0$ is a unique equilibrium point of [H]. Otherwise, any $z \in \mathbb{C}$ is an unstable equilibrium point.
        \item [(b)] 0 is table $\iff$ $\Gamma_{j}<1 ~\forall j$, where $\Gamma_{j} = \sqrt{a_{j}^{2} + b_{j}^{2}}$, $\lambda_{j} = a_{j} + i b_{j}$, $\lambda_{1}, ..., \lambda_{k} $ are roots of [CE].
    \end{itemize}
\end{theorem}
\begin{proof}
    \begin{enumerate}
        \item [(a)] $\tilde{x}$ is an equilibrium point $\iff$ $x_{n} = \tilde{x}$ for all $n$ \imp $\tilde{x} + a_1 \tilde{x} + ... + a_k \tilde{x} = 0$ \imp $\tilde{x}( 1+ a_1 + ... + a_k) = 0$
        \item [(b)] $n^{m_{j}} \Gamma_{j}^{n} (\cos(n \theta_{j}) - i \sin(n \theta_{j}))$ (General form of a basis function for [H])

        The conclusion follows from the argument in Theorem 6.3
    \end{enumerate}
\end{proof}

\begin{theorem}[Shur]
    ...
\end{theorem}

\section{Additional Topics}
\subsection*{Phase diagram}
$x_{n} = f(x_{n-2})$
\begin{center}
    \texttt{[insert a graph here]}
\end{center}

\begin{remark}
    $f'(\tilde{x}) < 1 \Longrightarrow \tilde{x}$ is locally asymptotically stable.

    $f'(\tilde{x}) > 1 \Longrightarrow \tilde{x}$ is unstable.    
\end{remark}

\subsection*{Linear Systems}
\[
\begin{cases}
    x_n = A_n x_{n-1} + B_{n} u_{n} & [C]\\
    x_n = A_n x_{n-1} & [H]    
\end{cases}
\]
Suppose $x_{0} = x^{0}$, then \[
\begin{aligned}
    x_{1} &= A_{1} x_{0} + B_1 u_1\\
    x_{2} &= A_{2} A_{1}x_{0} + A_{2} B_{1} u_{1} + B_{2} u_{2}\\
    &\vdots\\
    x_{n} &= \lb \prod_{s=0}^{n-1} A_{n-s} \rb x_{0} + \sum_{k=1}^{n}(\prod_{s=0}^{n-k-1} A_{n-s})B_{k} u_{k}
\end{aligned}
\]

\subsection*{Constant coefficients}
\[
    x_n = A x_{n-1} + B_{n} u_{n}  \Longrightarrow x_{n} = A^{n}x_{0} + \sum_{k=1}^{n}A^{n-k}B_{k}u_{k}
\]

\subsection*{Stability}
$x_{n} = A x_{n-1}$

$\tilde{x}$ is an equilibrium point $\iff$ $\tilde{x} = A\tilde{x}$.

\imp The set of equilibrium points is $H_{A}(1) = \{x \in \mathbb{C}^{n} \mid x = Ax\}$. (Eigenspace w.r.t. 1)

If 1 is not an eigenvalue of $A$, then $\tilde{x} = 0$ is a unique equilibrium point.

\begin{theorem}
    $\tilde{x}$ is a stable equilibrium point of [H] $\iff$ moduli of all eigenvalues of $A$ are less than 1.
\end{theorem}
\begin{proof}
    Note that $A = P D P^{-1}$ if diagonalizable.
\end{proof}

\begin{theorem}
    If $A \in \mathbb{R}^{n\times m}$, $A = (a_{ij}).$ Then $\sum_{j=1}^{n} |a_{ij}| < 1 \forall i$ \imp all eigenvalues with moduli $<1$. 
\end{theorem}

\begin{theorem}
    $x_n = f(x_{n-1})$, $f : \mathbb{C}^{n} \to \mathbb{C}^{n}$. Suppose $\tilde{x}$ is an equilibrium point, i.e. $\tilde{x} = f(\tilde{x}) $. if $Df(\tilde{x})$ has all eigenvalues with moduli <1, then $\tilde{x}$ is locally asymptotically stable.
\end{theorem}
\begin{proof}
    $x_{n} = f(x_{n-1}) \approx \underbrace{f(\tilde{x})}_{ = \tilde{x}} + Df(\tilde{x}) (x_{n-1} - \tilde{x}) $ \imp 
    $ \underbrace{x_{n} - \tilde{x}}_{y_{n}} \approx Df(\tilde{x}) \underbrace{(x_{n-1} - \tilde{x})}_{y_{n-1}} $.
\end{proof}










%=======================================

%$$##
%\clearpage
%\section*{References}
%\beginrefs
%\bibentry{CW87}{\sc D.~Coppersmith} and {\sc S.~Winograd}, 
%``Matrix multiplication via arithmetic progressions,''
%{\it Proceedings of the 19th ACM Symposium on Theory of %Computing},
%1987, pp.~1--6.
%\endrefs

% **** THIS ENDS THE EXAMPLES. DON'T DELETE THE FOLLOWING LINE:

\end{document}



\begin{theorem}
    Consider \begin{equation}
        x^{(n)} + a_{n-1} x^{(n-1)} + ... + a_1 x' + a_0 = 0 \qquad [H]
    \end{equation}
    0 is a stable equilibrium point of [H] $\Longrightarrow a_0, a_1, ..., a_{n-1} > 0$.
\end{theorem}
\begin{theorem}[Routh--Hurwitz]
    See the scanned page.
\end{theorem}

\section{System of linear ODE}
\begin{equation}
    \underbrace{\dot{x}(t)}_{n \times 1} = \underbrace{A(t)}_{n \times n} \underbrace{x(t)}_{n \times 1} + \underbrace{B(t)}_{n \times m} \underbrace{u(t)}_{m \times 1}
\end{equation}
\begin{remark}
    Consider 
    $$y^{(n)} + a_{n-1}(t)y^{(n-1)}+...+a_{1}(t)y' + a_{0}(t)y = g(t) $$
    Let 
    \[
    \begin{cases}
        x_1 &\equiv y\\
        x_2 &\equiv y'\\
        &\vdots\\
        x_n &\equiv y^{(n-1)}
    \end{cases}
    \Longrightarrow
    \begin{cases}
        \dot{x}_1 &\equiv x_2\\
        \dot{x}_2 &\equiv x_3\\
        &\vdots\\
        \dot{x}_{n-1} &\equiv x_n
    \end{cases}
    \]

    \[  \Longrightarrow
        \dot{x}_n + a_{n-1}(t)x_n+...+a_{1}(t)x_2 + a_{0}(t)x_1 = g(t)
    \]

    \[
    x(t) = \begin{bmatrix}
       x_1(t) \\[0.3em]
       \vdots \\[0.3em]
       x_n(t)      
     \end{bmatrix}
     \qquad
    u(t) = \begin{bmatrix}
       u_1(t) \\[0.3em]
       \vdots \\[0.3em]
       u_m(t)      
     \end{bmatrix}
    \]

    \[
    A(t) = \begin{bmatrix}
       a_{11}(t)& \cdots & a_{1n}(t)    \\[0.3em]
       \vdots   &        & \vdots       \\[0.3em]
       a_{n1}(t)& \cdots & a_{nn}(t)    
     \end{bmatrix}
     \qquad
    B(t) = \begin{bmatrix}
       b_{11}(t)& \cdots & b_{1m}(t)    \\[0.3em]
       \vdots   &        & \vdots       \\[0.3em]
       b_{n1}(t)& \cdots & b_{nm}(t)    
     \end{bmatrix}
    \]
    Then we have
    \begin{equation}
    \underbrace{\begin{bmatrix}
            \dot{x}_1(t) \\[0.3em]
            \dot{x}_2(t) \\[0.3em]
            \vdots \\[0.3em]
            \dot{x}_{n-1}(t) \\[0.3em]
            \dot{x}_n(t)      
         \end{bmatrix}}_{\dot{x}(t)} 
    = \underbrace{\begin{bmatrix}
            0 & 1& 0& \cdots   & 0 & 0   \\[0.3em]
            0 & 0& 1& \cdots   & 0 & 0   \\[0.3em]
            \vdots & \vdots   & \vdots   &   \ddots     & \vdots& \vdots       \\[0.3em]
            0 & 0& 0& \cdots   & 0 & 1   \\[0.3em]
            -a_0(t) & -a_{1}(t)& -a_{2}(t)& \cdots & -a_{n-2}(t)& -a_{n-1}(t)    
          \end{bmatrix}}_{A(t)}
    \underbrace{\begin{bmatrix}
            x_1(t) \\[0.3em]
            x_2(t) \\[0.3em]
            \vdots \\[0.3em]
            x_{n-1}(t) \\[0.3em]
            x_n(t)      
         \end{bmatrix}}_{x(t)} + 
    \underbrace{\begin{bmatrix}
            0 \\[0.3em]
            0 \\[0.3em]
            \vdots \\[0.3em]
            0 \\[0.3em]
            1         
        \end{bmatrix} }_{B(t)} \underbrace{g(t)}_{u(t)}
    \end{equation}
\end{remark}

\begin{theorem}
    If all elements of $A(t), B(t), u(t)$ are continuous, then $\forall$ initial condition $x(t_{0}) = x^{0}$, $\exists$ a unique solution to [C]
\end{theorem}

\begin{remark}  
    \begin{equation}
        \dot{x}(t) = A(t)x(t)+ B(t)u(t) \qquad [C]
    \end{equation}
    \begin{equation}
        \dot{x}(t) = A(t)x(t)  \qquad [H]
    \end{equation}
\end{remark}

\begin{theorem}
    $\lcb H \rcb$ is a vector space.
\end{theorem}
\begin{proof}
    Excercise.
\end{proof}

\begin{theorem}
    $\dim \lcb H \rcb = n$.
\end{theorem}

\begin{definition}
    Let $\lcb x^1(t), ..., x^n(t) \rcb$ be a basis of $\lcb H \rcb$. Then \begin{equation}
        \Phi(t) = (x^1(t), ..., x^n(t)) = \begin{bmatrix}
            x_{1}^{1}(t)& \cdots & x_{1}^{n}(t)    \\[0.3em]
       \vdots   &        & \vdots       \\[0.3em]
       x_{n}^{1}(t)& \cdots & x_{n}^{n}(t)  
        \end{bmatrix}
    \end{equation}
    where $\Phi(t)$ is a fundamental matrix of [H].
\end{definition}

\section{Systems of linear ODE}
\begin{remark}
    \begin{enumerate}
        \item There are infinitely many fundamental matrices of [H]
        \item $\dot{\Phi}(t) \equiv (\dot{x}^1(t), ..., \dot{x}^n(t)) 
        = \begin{bmatrix}
            \dot{x}_{1}^{1}(t)& \cdots & \dot{x}_{1}^{n}(t)    \\[0.3em]
            \vdots   &        & \vdots       \\[0.3em]
            \dot{x}_{n}^{1}(t)& \cdots & \dot{x}_{n}^{n}(t)  
        \end{bmatrix}$
    \end{enumerate}
\end{remark}

\begin{lemma}
    $\Phi(t)$ is a fundamental matrix of [H] $\iff$ $\dot{\Phi}(t) = A(t)\Phi(t)$ and $\Phi(t)$ is invertible for all $t$.
\end{lemma}
\begin{proof}
    ``\imp''
    \begin{equation}
        \dot{\Phi}(t) = (\dot{x}^1(t), ..., \dot{x}^n(t)) = (A(t){x}^1(t), ..., A(t){x}^n(t)) = A(t) (x^1(t), ..., x^n(t)) = A(t) \Phi(t)
    \end{equation}
    ``$\Longleftarrow$''

    $\Phi(t)$ has linearly independent columns. Suppose $x(t)$ is a solution to [H] that satisfies $x(t_0) = x^{0}$. Consider $\tilde{x}(t) = \Phi(t)c$. Then $\tilde{x}(t_0) = \Phi(t_0) c \Longrightarrow c = \Phi(t_0)^{-1}\tilde{x}(t_0) $. 

    Suppose $\tilde{x}(t_0) = x^0$. Then $\tilde{x}(t) = \Phi(t) \Phi(t_0)^{-1} x^0 $. Then
    \begin{equation}
        \dot{\tilde{x}}(t) = \dot{\Phi}(t)\Phi(t_0)^{-1} x^0 = A(t) \underbrace{\Phi(t) \Phi(t_0)^{-1} x^0}_{\tilde{x}(t)} = A(t) \tilde{x}(t)
    \end{equation}
    Hence $\tilde{x}(t)$ is a solution to [H]. By uniqueness, \begin{equation}
        \tilde{x}(t) = x(t) = \Phi(t) \underbrace{\Phi(t_0)^{-1}x^0}_{c}
    \end{equation}
    Hence, $\Phi(t)$ has a basis of $\{ H \}$ as columns \imp fundamental matrix. 
\end{proof}


\begin{definition}
    $\Phi(t,t_{0})$ is a \underline{state transition matrix of } [H] if 
    \begin{enumerate}
        \item $\dot{\Phi}(t,t_0) = A(t) {\Phi}(t,t_0)$
        \item ${\Phi}(t_0,t_0) = I_{n}$
    \end{enumerate}
\end{definition}

\begin{theorem}
    If $\Phi(t)$ is a fundamental matrx of [H], then $\Phi(t,t_{0}) = \Phi(t)\Phi(t_0)^{-1}$ is a state transition matrx.
\end{theorem}
\begin{proof}
\begin{enumerate}
    \item $\dot{\Phi}(t,t_0) =  \dot{\Phi}(t)\Phi(t_0)^{-1} = A(t) {\Phi}(t) {\Phi}(t_0) = A(t) {\Phi}(t,t_0)$
    \item ${\Phi}(t_0,t_0) = {\Phi}(t_0){\Phi}(t_0)^{-1} = I_n$
\end{enumerate}
\end{proof}

\textbf{Remark}: State transition matrx is unique.
\begin{proof}
    Exercise.
\end{proof}

\begin{theorem}
    Suppose assumptions of theorem ???(7.1) holds. Then $x(t) = \Phi(t,t_0)x^0$ is the unique solution to [H] that satisfies $x(t_0) = x^{0}$.
\end{theorem}
\begin{proof}
    $x(t) = \Phi(t)c$ for some $c \in \mathbb{R}^n$ \imp $x(t_0) = \Phi(t_0)c$ \imp $c = \Phi(t_0)^{-1} x(t_{0})$

    \imp $x(t)= \Phi(t)\Phi(t_0)^{-1}x(t_0) = \Phi(t,t_0)x(t_0)$
\end{proof}

\begin{remark}
    $x(t) = \Phi(t) c, ~c \in \mathbb{R}^{n}$ is the general solution to [H].
\end{remark}

\begin{theorem}
    The general solution to [C] is 
    \begin{equation}
        x(t) = \Phi(t) c + \int_{t_{0}}^{t} \Phi(t,s) B(s) u(s) \dd s.
    \end{equation}
    Hence, the solution that satisfies $x_{t_{0}} = x^{0}$ is \begin{equation}
        x(t) = \Phi(t,t_{0}) x^{0} + \int_{t_{0}}^{t} \Phi(t,s) B(s) u(s) \dd s
    \end{equation}
\end{theorem}

\section{Systems of linear ODE with constant coefficients}
\begin{equation}
    \dot{x}(t) = A x(t) + B(t)u(t) \qquad [C]
\end{equation}
\begin{equation}
    \dot{x}(t) = A x(t) \qquad [H]
\end{equation}
\begin{definition}
    Let $A \in \mathbb{R}^{n\times n}$. Then $e^{A} \equiv \sum_{k=0}^{\infty} \frac{A^{k}}{k!}$ is the \underline{matrix exponential} of A.
\end{definition}
\begin{theorem}
    Let $A, B, C \in \mathbb{R}^{n\times n}$, and $c, d \in \mathbb{R}$.
    \begin{enumerate}
        \item Suppose $AB = BA$. Then \[
        e^{cA}e^{dB} = e^{cA+dB}
        \]
        In particular, $e^{cA}e^{dA} = e^{(c+d)A}$.
        \item $\det(e^{A}) = e^{tr{(A)}}$. Hence, $e^{A}$ is invertible $\forall A \in \mathbb{R}^{n \times n}$.
        \item Suppose $D$ is diagonal, i.e. \[
        D = diag(d_{i}) = \begin{bmatrix}
            d_{1} & & 0\\
            &\ddots& \\
            0 & & d_{n}
        \end{bmatrix}
        \]
        Then \[
        e^{D} = diag(e^{d_{i}}) = \begin{bmatrix}
            e^{d_{1}} & & 0\\
            &\ddots& \\
            0 & & e^{d_{n}}
        \end{bmatrix}.
        \]
        In particular, $e^{0_{n \times n}} = I_{n}, ~e^{I_{n}} = e I_{n}$.
        \item $(e^{A})^{-1} = e^{-A}$.
        \item Let $A = P J P^{-1}$, where $J$ -- Jordan normal form, then \[
        e^{A} = P e^{J} P^{-1}.
        \]
        Suppose $A$ is diagonalizable, $A = P D P^{-1}$, then $e^{A} = P e^{D} P^{-1}$
        \item  $
        \dfrac{\dd }{\dd t} \lb e^{At} \rb = A e^{At}.
        $
    \end{enumerate}
\end{theorem}
\begin{proof}
    \begin{itemize}
        \item 1. follows from Binomial formula / Cauchy product.
        \item 2. follows from Jacobi's formula.
        \item 3. -- 6. Exercise.
        \item For 4., use 1.
    \end{itemize}
\end{proof}

\section{Systems of Linear ODE with Constant coeffiicients}
\begin{theorem}
    \begin{equation}
    \begin{cases}
        \dot{x}(t) = A(t)x(t)+ B(t)u(t) \qquad [C]\\
        \dot{x}(t) = A(t)x(t)  \qquad [H]
    \end{cases}
    \end{equation}
    Then $e^{At}$ is a fundamental matrix of [H]. $e^{A(t-t_0)}$ is a state transtition matrix.
\end{theorem}
\begin{proof}
    $e^{At}$ is invertible, $(e^{At})^{-1} = e^{-At}$. And $\frac{\dd}{\dd t}(e^{At}) = A e^{At}$. By Theorem 6.9, we know $e^{A(t-t_0)} = e^{At} (e^{At_{0}})^{-1}$ is a state transtition matrix.
\end{proof}

\begin{theorem}
    General solution to [H] is $x(t) = e^{A(t-t_{0})}x(t_{0})  $.

    General solution to [C] is $x(t) = e^{A(t-t_{0})}x(t_{0}) + \int_{t_{0}}^{t} e^{A(t-s)}B(s)u(s) \dd s$.
\end{theorem}

\textbf{Exercise}: Find the general solution of \[
    \begin{bmatrix}
        \dot{x}_{1}(t)\\
        \dot{x}_{2}(t)
    \end{bmatrix}
    =
    \begin{bmatrix}
        a_{1} & 0\\
        0 & a_{2}
    \end{bmatrix}
    \begin{bmatrix}
        {x}_{1}(t)\\
        {x}_{2}(t)
    \end{bmatrix}
    +
    \begin{bmatrix}
        1 & 0\\
        0 & 1
    \end{bmatrix}
    \begin{bmatrix}
        u_{1}(t)\\
        u_{2}(t)
    \end{bmatrix}
\]

\section{Stability of Systems of ODE}
\begin{theorem}
    Consider a linear system $\dot{x}(t) = A x (t)$
    \begin{enumerate}
        \item [(a)] 0 is a unique equilibrium point $\iff$ $\det A \neq 0$
        \item [(b)] 0 is a stable equilibrium point $\iff$ all eigenvalues of $A$ have negative real parts.
    \end{enumerate}
\end{theorem}
\begin{proof}
    (a) Suppose $Ax = 0$ 

    ``\imp'' part: $Ax \neq 0$ for any $x \neq 0$ \imp $\det(A) \neq 0$. Contradiction!

    ``$\Longleftarrow$'' part: $Ax = 0$ \imp $x=0$ is the unique equilibrium point.

    (b) Suppose $A$ is diagonalizable [if not, a similar argument can be developed using Jordan decomposition]. So $A = P D P^{-1}$ \imp $\dot{x} = P D P^{-1} x $ \imp $\underbrace{P^{-1} \dot{x}}_{y} = D \underbrace{ P^{-1} x}_{\dot{y}}$ $\iff$ $\dot{y} = D y$.

    Note that $\lim_{t \to \infty} x(t) = 0 \iff \lim_{t \to \infty} y(t) = 0 $.

    Now look at $\dot{y} = D y$:
    $$\dot{y} = \begin{bmatrix}
        \dot{y}_{1}\\
        \vdots\\
        \dot{y}_{n}
    \end{bmatrix} = 
    \begin{bmatrix}
        d_{1} & &0 \\
            & \ddots &\\
        0&& d_{n}
    \end{bmatrix}
    \begin{bmatrix}
        y_{1}\\
        \vdots\\
        y_{n}
    \end{bmatrix}
    \iff 
    \dot{y}_{j} = d_{j} y_{j}, ~ j = 1, ..., n.$$
    $y_{j}(t) = c_{j} e^{\alpha_{t} t} = c_{j} e^{(a_{t}+ i b_{j}) t}.$
    Then $\lim_t y_{j}(t) = 0 \iff a_{j} < 0 
    $. Hence the conclusion follows.
\end{proof}

\begin{theorem}
    Consider a non-linear system $\dot{x} = f(x), ~ f : \mathbb{Q}^{n} \to \mathbb{R}^{n}$. Suppose $f$ is $C^{1}$ and $\tilde{x}$ is an equilibrium point. Then
    \begin{itemize}
        \item [(a)] all eigenvalues of $Df(\tilde{x})$ havenegative real parts \imp $\tilde{x}$ is locally asymptotically stable.
        \item [(b)] at least one eigenvalue of  $Df(\tilde{x})$ has positive real part \imp $\tilde{x}$ is unstable.
    \end{itemize}
\end{theorem}
\begin{proof}
    (For a rigorous argument, see Hartman-Grobman theorem.)

    By Taylor theorem,
    \[
    \underbrace{f(x)}_{\dot{x}} = \underbrace{f(\tilde{x})}_{=0}+ Df(\tilde{x})(x-\tilde{x}) + \Gamma (x - \tilde{x})
    \]
    and $\lim_{x\to \tilde{x}} \frac{\Gamma (x - \tilde{x})}{||x - \tilde{x}||} = 0$.
    \[
    \Longrightarrow \dot{x} = D f (\tilde{x}) (x- \tilde{x}) + \Gamma (x-\tilde{x})
    \]
    \[
    \iff \frac{\dd}{\dd t}(x - \tilde{x}) = Df(\tilde{x})\underbrace{(x-\tilde{x})}_{y} + \Gamma (x- \tilde{x})
    \]
    \[
    \Longrightarrow \dot{y} = D f(\tilde{x}) y + \Gamma (y)
    \]
    Then when $x$ is close to $\tilde{x}$, or $y $ is close to $0$, the stability of the original system is determined by the stability of its linearized version: $\dot{y} = Df(\tilde{x})y$. The conclusion follows from Theorem 7.3.
\end{proof}

\subsection*{Phase Diagrams}
\begin{equation}
    \begin{cases}
        \dot{x_{1}} = f_{1}(x_{1}, x_{2})\\
        \dot{x_{2}} = f_{2}(x_{1}, x_{2})
    \end{cases}
\end{equation}
\begin{example}
    \begin{equation}
    \begin{cases}
        \dot{x_{1}} = x_{2}\\
        \dot{x_{2}} = x_{1}
    \end{cases} \iff
    \begin{bmatrix}
        \dot{x_{1}} \\
        \dot{x_{2}} 
    \end{bmatrix} = 
    \begin{bmatrix}
        0&1\\
        1&0
    \end{bmatrix}
    \begin{bmatrix}
        \dot{x_{1}} \\
        \dot{x_{2}} 
    \end{bmatrix}
\end{equation}
\begin{center}
    \texttt{[Insert a phase diagram here]}
\end{center}
\end{example}
\begin{remark}
    Saddle path correspondences to an eigenvector with eigenvalue that has negative real part.
\end{remark}



