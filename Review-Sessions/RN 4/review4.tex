% See instructions in the preamble file
\documentclass[twoside]{article}

%
% This is a borrowed LaTeX template file for lecture notes for CS267,
% Applications of Parallel Computing, UCBerkeley EECS Department.
% Now being used for CMU's 10725 Fall 2012 Optimization course
% taught by Geoff Gordon and Ryan Tibshirani.  When preparing 
% LaTeX notes for this class, please use this template.
%
% To familiarize yourself with this template, the body contains
% some examples of its use.  Look them over.  Then you can
% run LaTeX on this file.  After you have LaTeXed this file then
% you can look over the result either by printing it out with
% dvips or using xdvi. "pdflatex template.tex" should also work.
%

\setlength{\oddsidemargin}{0.25 in}
\setlength{\evensidemargin}{-0.25 in}
\setlength{\topmargin}{-0.6 in}
\setlength{\textwidth}{6.5 in}
\setlength{\textheight}{8.5 in}
\setlength{\headsep}{0.75 in}
\setlength{\parindent}{0 in}
\setlength{\parskip}{0.1 in}

%
% ADD PACKAGES here:
%

\usepackage{amsmath,amsfonts,graphicx}

%
% The following commands set up the lecnum (lecture number)
% counter and make various numbering schemes work relative
% to the lecture number.
%
\newcounter{lecnum}
\renewcommand{\thepage}{\thelecnum-\arabic{page}}
\renewcommand{\thesection}{\thelecnum.\arabic{section}}
\renewcommand{\theequation}{\thelecnum.\arabic{equation}}
\renewcommand{\thefigure}{\thelecnum.\arabic{figure}}
\renewcommand{\thetable}{\thelecnum.\arabic{table}}

%
% The following macro is used to generate the header.
%
\newcommand{\lecture}[4]{
   \pagestyle{myheadings}
   \thispagestyle{plain}
   \newpage
   \setcounter{lecnum}{#1}
   \setcounter{page}{1}
   \noindent
   \begin{center}
   \framebox{
      \vbox{\vspace{2mm}
    \hbox to 6.28in { {\bf Econ 626: Quantitative Methods II
  \hfill Fall 2018} }
       \vspace{4mm}
       \hbox to 6.28in { {\Large \hfill Lecture #1: #2  \hfill} }
       \vspace{2mm}
       \hbox to 6.28in { {\it Lecturer: #3 \hfill Scribes: #4} }
      \vspace{2mm}}
   }
   \end{center}
   \markboth{Lecture #1: #2}{Lecture #1: #2}

   %{\bf Note}: {\it LaTeX template courtesy of UC Berkeley EECS dept.}

   {\bf Disclaimer}: {\it Zhikun is fully responsible for the errors and typos appeared in the notes.}
   \vspace*{4mm}
}
%
% Convention for citations is authors' initials followed by the year.
% For example, to cite a paper by Leighton and Maggs you would type
% \cite{LM89}, and to cite a paper by Strassen you would type \cite{S69}.
% (To avoid bibliography problems, for now we redefine the \cite command.)
% Also commands that create a suitable format for the reference list.
\renewcommand{\cite}[1]{[#1]}
\def\beginrefs{\begin{list}%
        {[\arabic{equation}]}{\usecounter{equation}
         \setlength{\leftmargin}{2.0truecm}\setlength{\labelsep}{0.4truecm}%
         \setlength{\labelwidth}{1.6truecm}}}
\def\endrefs{\end{list}}
\def\bibentry#1{\item[\hbox{[#1]}]}

%Use this command for a figure; it puts a figure in wherever you want it.
%usage: \fig{NUMBER}{SPACE-IN-INCHES}{CAPTION}
\newcommand{\fig}[3]{
      \vspace{#2}
      \begin{center}
      Figure \thelecnum.#1:~#3
      \end{center}
  }
% Use these for theorems, lemmas, proofs, etc.
\newtheorem{theorem}{Theorem}[lecnum]
\newtheorem{lemma}[theorem]{Lemma}
\newtheorem{proposition}[theorem]{Proposition}
\newtheorem{claim}[theorem]{Claim}
\newtheorem{corollary}[theorem]{Corollary}
\newtheorem{definition}[theorem]{Definition}
%\newtheorem{example}[theorem]{Example}
\newenvironment{proof}{{\bf Proof:}}{\hfill\rule{2mm}{2mm}}
\newenvironment{example}{{\bf Example:}}{\hfill\rule{2mm}{2mm}}

\newtheorem{remark}[theorem]{Remark}
%\newenvironment{remark}[1][Remark]{\begin{trivlist}\item[\hskip \labelsep {\bfseries #1}]}{\end{trivlist}}


% **** IF YOU WANT TO DEFINE ADDITIONAL MACROS FOR YOURSELF, PUT THEM HERE:

\newcommand\E{\mathbb{E}}
\newcommand\dd{\mathrm{d}}

\usepackage{hyperref}
\usepackage{cancel}
\usepackage{listings}

\newcommand\pp{\partial}
\newcommand\imp{$\Longrightarrow$}

\begin{document}

%\lecture{**LECTURE-NUMBER**}{**DATE**}{**LECTURER**}{**SCRIBE**}
\lecture{4}{Review Session \#4}{Aliaksandr Zaretski}{Zhikun Lu}
\footnotetext[1]{Visit \url{http://www.luzk.net/misc} for updates.}

%\tableofcontents
\section{Complex Numbers} 

$\mathbb{C}$ -- the set of Complex numbers. 

If $z\in \mathbb{C}$, then $Z = a+b i $, where $a$ is the real part and $b$ is the imaginary part.

\underline{Polar form}: We use $|z| = \sqrt{a^2+b^2}$ -- modulus. Then $z = |z|e^{i \theta} = |z|({\cos \theta + i \sin \theta})$.

If $\theta = \pi$, then $e^{i\pi} = -1$ (Euler's identity).

\begin{equation}
    \Longrightarrow
    \begin{cases}
            a = |z| \cos \theta \\
            b = |z| \sin \theta     
    \end{cases}    \Longrightarrow \frac{b}{a} = \tan \theta
\end{equation}

\begin{remark}\label{rem:a}
    To find $\theta$ based on $a$, $b$, use the $atan2$ function:
    \begin{equation}
        \theta = atan2(b,a) = \begin{cases}
            atan(\frac{b}{a}), &\text{ if }a>0\\
            atan(\frac{b}{a})+\pi, &\text{ if }a<0 \wedge b\geq 0\\
            atan(\frac{b}{a})- \pi, &\text{ if }a<0 \wedge b <0\\
            \frac{\pi}{2}, &\text{ if }a=0 \wedge b>0\\
            -\frac{\pi}{2}, &\text{ if }a=0 \wedge b<0\\
            \text{undefined}, &\text{ if }a=0 \wedge b=0   
        \end{cases}
    \end{equation}
\end{remark}

\section{Solutions to the characteristic equation [CE]}
Suppose $\lambda \in \mathbb{C}$ is a solution to [CE]. Then $\lambda = a + b i$. Then $e^{\lambda t}$ is a solution to [H].
\begin{equation}
    e^{\lambda t} = e^{(a+bi)t} = e^{at} e^{ibt} = e^{a t}(\cos (bt) + i \sin (bt))
\end{equation}
Also, $\bar{\lambda} = a - bi$ is a solution to [CE].
\begin{equation}
    e^{\bar{\lambda}} = e^{(a-bi)t} =  e^{a t}(\cos (-bt) + i \sin (-bt)) = e^{a t}(\cos (bt) - i \sin (bt))
\end{equation}
Consider $ (e^{\lambda t}, e^{\bar{\lambda} t}) $ -- these are two independent solutions to [H]. Their linear combination
\begin{eqnarray}
    C_1 e^{\lambda t} + C_2 e^{\bar{\lambda} t} &=& e^{at}[(C_1 + C_2) \cos (bt) + i (C_1 - C_2) \sin (bt)]\\
    &=& e^{at}[C_3 \cos (bt) + C_4 \sin (bt)], \quad C_3, C_4 \in \mathbb{C}
\end{eqnarray}

\section{Related solutions to [CE]}
Suppose $\mu(\lambda) = m > 1$. Then the independent solutions associated to $\lambda$ are $\{ e^{\lambda t}, t e^{\lambda t}, ..., t^{m-1} e^{\lambda t} \}$.

\underline{Intuition from 2nd order case}
$$x'' + a_1 x' + a_0 x = 0 \qquad \text{ [H] }$$
$$\lambda^2 + a_1 \lambda + a_0 = 0 \qquad \text{ [CE] }$$
$\mathcal{D} = a_1^2 - 4 a_0$
\begin{equation}
    \begin{cases}
        \mathcal{D} > 0 &\Longrightarrow \text{ distinct real roots }\\
        \mathcal{D} = 0 &\Longrightarrow \text{ equal real roots }\\
        \mathcal{D} < 0 &\Longrightarrow \text{ complex conjugates }
    \end{cases}
\end{equation}
Suppose $\mathcal{D} = 0$. Then  $\lambda_{1,2} = \frac{-a_1}{2}$. We have solutions $e^{\lambda t}$, $t e^{\lambda t}$.

Consider $x(t) = t e^{\lambda t}$, 
\begin{equation}
    x'(t) = e^{\lambda t} + \lambda t e^{\lambda t}
\end{equation}
\begin{equation}
    x''(t) = 2 \lambda e^{\lambda t} + \lambda^2 t e^{\lambda t}
\end{equation}
Substitute into [H]
\begin{eqnarray}
    2\lambda e^{\lambda t} + \lambda^2 t e^{\lambda t} + a_1 e^{\lambda t} + a_1 \lambda t e^{\lambda t} + a_0 t e^{\lambda t} = 0\\
    2\lambda  + \lambda^2 t  + a_1  + a_1 \lambda t  + a_0 t = 0\\
    2(\frac{-a_1}{2})  + (\frac{-a_1}{2})^2 t  + a_1  + a_1 (\frac{-a_1}{2}) t  + a_0 t = 0\\
     (a_1^2 - 4 a_0)t = 0
\end{eqnarray}

\underline{General solution to [H]}

Consider, for example, the following:
$$x^{(9)} + a_8 x^{(8)} +...+ a_1 x'+ a_0 = 0 \qquad \text{ [H] }$$
$$\lambda^9 + a_8 \lambda^8 + ... + a_1 \lambda + a_0 = 0 \qquad \text{ [CE] }$$
Let $\lambda_1, ..., \lambda_9$ be the roots of [CE]. Suppose $\lambda_1, \lambda_2, ...,\lambda_5 \in \mathbb{R}$, $\lambda_6, ..., \lambda_9 \in \mathbb{C}$. Suppose 
\begin{equation}
    \begin{cases}
        \lambda_1 \neq \lambda_2 \neq \lambda_3\\
        \lambda_3 = \lambda_4 = \lambda_5\\
        \lambda_6 = \lambda_8\\
        \lambda_7 = \lambda_9
    \end{cases}
\end{equation}
Let $\lambda_6 = a+ b i$. Then $\lambda_7 = a-bi, \lambda_8 = a+bi, \lambda_9 = a-bi$.

Then the general solution to [H] is 
\begin{equation}
    x_n(t) = C_1 e^{\lambda_1 t} + C_2 e^{\lambda_2 t} + C_3 e^{\lambda_3 t} + C_4 e^{\lambda_4 t}+ C_5 e^{\lambda_5 t} + e^{at}(C_6 \cos(bt)+ C_7 \sin(bt)) + t e^{at}(C_8 \cos(bt)+ C_9 \sin(bt))
\end{equation}

\underline{Particular solution to [C]}
\begin{equation}
    Lx(t) = g(t) \qquad \text{[C]}
\end{equation}

\begin{table}[htbp]
\begin{tabular}{ll}
$g(t)$ & Guess for $x_p(t)$ \\ \hline
$C$ -- constant  & $D$ -- constant \\
$e^{\lambda t}$ & $D e^{\lambda t}$   \\
$t^\Gamma$ & $b_0+b_1 t + ... + b_\Gamma t^\Gamma$ \\
$\sin(bt)$ & $D\sin(bt)+E\cos(bt)$ \\
$\cos(bt)$ & $D\sin(bt)+E\cos(bt)$ \\
$\lambda^t$ & $D \lambda^t$ \\
sum or product & sum or product \\
 of the above &  of the above  
\end{tabular}
\end{table}

\begin{remark}
    If the guess for $x_p(t)$ solves [H], then multiply the guess by $t$.
\end{remark}

\begin{example}
    \begin{equation}
        \begin{cases}
            x'' + 2x' +2x = t^2, &\text{ [C] }\\
            x'' + 2x' +2x = 0  , &\text{ [H] }\\
            \lambda^2 + 2 \lambda +2 = 0  , &\text{ [CE] }
        \end{cases}
    \end{equation}
    $\mathcal{D} = 4 - 8 < 0 $, $\lambda_{1,2} = -1 \pm i
    $. So $a = -1, b=1$
    \begin{equation}
        x_h(t) = e^{-t} (C_1 \cos(t) + C_2 \sin(t))  \text{ -- general solution to [H]}
    \end{equation}
    Try
    \begin{equation}
        x_p(t) = b_0 + b_1 t + b_2 t^2 \Longrightarrow \begin{cases}
            x_p' = b_1 + 2b_2 t\\
            x_p'' = 2 b_2
        \end{cases}
    \end{equation}
    Substitute into [C]:\begin{equation}
        (2 b_2) + 2 ( b_1 + 2b_2 t ) + 2 (b_0 + b_1 t + b_2 t^2 ) = t^2
    \end{equation}
    We need: $\begin{cases}
        2 b_2 = 1 &\Longrightarrow b_2 = \frac{1}{2}\\
        4 b_2 + 2 b_1 = 0 &\Longrightarrow b_1 = -1 \\
        2 b_2 + 2 b_1 + 2 b_0= 0 &\Longrightarrow b_0 = \frac{1}{2}
    \end{cases}$
    $$\Longrightarrow x_p(t) = \frac{1}{2} - t + \frac{1}{2}t^2 \qquad \text{-- particular solution tp [C]}$$
    Then, the general solution to [C] is \begin{equation}
        x(t) = e^{-t} (C_1 \cos(t) + C_2 \sin(t)) + \frac{1}{2} - t + \frac{1}{2}t^2
    \end{equation}
    Now suppose we have initial conditions $\begin{cases}
        x(0) = \frac{1}{2}\\
        x'(0) = 0
    \end{cases}$

    Notice that $x'(t) = -e^{-t} (C_1 \cos(t) + C_2 \sin(t)) + e^{-t} ( - C_1 \sin(t) + C_2 \cos(t))- 1 + t$

    \begin{equation}
    \begin{cases}
        x(0) = C_1 + \frac{1}{2} = \frac{1}{2} &\Longrightarrow C_1 = 0\\
        x'(0) = -C_1 + C_2 -1 =0 &\Longrightarrow C_2 = 1
    \end{cases}
    \end{equation}
    Then the particular solution that satisfies $\begin{cases}
        x(0) = \frac{1}{2}\\
        x'(0) = 0
    \end{cases}$ is \begin{equation}
        x_p(t) = e^{-t}\sin(t) + \frac{1}{2} - t + \frac{1}{2} t^2
    \end{equation}


\end{example}










%=======================================































%=======================================

%$$##
\clearpage
\section*{References}
%\beginrefs
%\bibentry{CW87}{\sc D.~Coppersmith} and {\sc S.~Winograd}, 
%``Matrix multiplication via arithmetic progressions,''
%{\it Proceedings of the 19th ACM Symposium on Theory of %Computing},
%1987, pp.~1--6.
%\endrefs

% **** THIS ENDS THE EXAMPLES. DON'T DELETE THE FOLLOWING LINE:

\end{document}





