% See instructions in the preamble file
\documentclass[twoside]{article}

%
% This is a borrowed LaTeX template file for lecture notes for CS267,
% Applications of Parallel Computing, UCBerkeley EECS Department.
% Now being used for CMU's 10725 Fall 2012 Optimization course
% taught by Geoff Gordon and Ryan Tibshirani.  When preparing 
% LaTeX notes for this class, please use this template.
%
% To familiarize yourself with this template, the body contains
% some examples of its use.  Look them over.  Then you can
% run LaTeX on this file.  After you have LaTeXed this file then
% you can look over the result either by printing it out with
% dvips or using xdvi. "pdflatex template.tex" should also work.
%

\setlength{\oddsidemargin}{0.25 in}
\setlength{\evensidemargin}{-0.25 in}
\setlength{\topmargin}{-0.6 in}
\setlength{\textwidth}{6.5 in}
\setlength{\textheight}{8.5 in}
\setlength{\headsep}{0.75 in}
\setlength{\parindent}{0 in}
\setlength{\parskip}{0.1 in}

%
% ADD PACKAGES here:
%

\usepackage{amsmath,amsfonts,graphicx}

%
% The following commands set up the lecnum (lecture number)
% counter and make various numbering schemes work relative
% to the lecture number.
%
\newcounter{lecnum}
\renewcommand{\thepage}{\thelecnum-\arabic{page}}
\renewcommand{\thesection}{\thelecnum.\arabic{section}}
\renewcommand{\theequation}{\thelecnum.\arabic{equation}}
\renewcommand{\thefigure}{\thelecnum.\arabic{figure}}
\renewcommand{\thetable}{\thelecnum.\arabic{table}}

%
% The following macro is used to generate the header.
%
\newcommand{\lecture}[4]{
   \pagestyle{myheadings}
   \thispagestyle{plain}
   \newpage
   \setcounter{lecnum}{#1}
   \setcounter{page}{1}
   \noindent
   \begin{center}
   \framebox{
      \vbox{\vspace{2mm}
    \hbox to 6.28in { {\bf Econ 626: Quantitative Methods II
  \hfill Fall 2018} }
       \vspace{4mm}
       \hbox to 6.28in { {\Large \hfill Lecture #1: #2  \hfill} }
       \vspace{2mm}
       \hbox to 6.28in { {\it Lecturer: #3 \hfill Scribes: #4} }
      \vspace{2mm}}
   }
   \end{center}
   \markboth{Lecture #1: #2}{Lecture #1: #2}

   %{\bf Note}: {\it LaTeX template courtesy of UC Berkeley EECS dept.}

   {\bf Disclaimer}: {\it Zhikun is fully responsible for the errors and typos appeared in the notes.}
   \vspace*{4mm}
}
%
% Convention for citations is authors' initials followed by the year.
% For example, to cite a paper by Leighton and Maggs you would type
% \cite{LM89}, and to cite a paper by Strassen you would type \cite{S69}.
% (To avoid bibliography problems, for now we redefine the \cite command.)
% Also commands that create a suitable format for the reference list.
\renewcommand{\cite}[1]{[#1]}
\def\beginrefs{\begin{list}%
        {[\arabic{equation}]}{\usecounter{equation}
         \setlength{\leftmargin}{2.0truecm}\setlength{\labelsep}{0.4truecm}%
         \setlength{\labelwidth}{1.6truecm}}}
\def\endrefs{\end{list}}
\def\bibentry#1{\item[\hbox{[#1]}]}

%Use this command for a figure; it puts a figure in wherever you want it.
%usage: \fig{NUMBER}{SPACE-IN-INCHES}{CAPTION}
\newcommand{\fig}[3]{
      \vspace{#2}
      \begin{center}
      Figure \thelecnum.#1:~#3
      \end{center}
  }
% Use these for theorems, lemmas, proofs, etc.
\newtheorem{theorem}{Theorem}[lecnum]
\newtheorem{lemma}[theorem]{Lemma}
\newtheorem{proposition}[theorem]{Proposition}
\newtheorem{claim}[theorem]{Claim}
\newtheorem{corollary}[theorem]{Corollary}
\newtheorem{definition}[theorem]{Definition}
%\newtheorem{example}[theorem]{Example}
\newenvironment{proof}{{\bf Proof:}}{\hfill\rule{2mm}{2mm}}
\newenvironment{example}{{\bf Example:}}{\hfill\rule{2mm}{2mm}}

\newenvironment{remark}{{\bf Remark:}}{\hfill\rule{2mm}{2mm}}
%\newtheorem{remark}[theorem]{Remark}

%\newenvironment{remark}[1][Remark]{\begin{trivlist}\item[\hskip \labelsep {\bfseries #1}]}{\end{trivlist}}


% **** IF YOU WANT TO DEFINE ADDITIONAL MACROS FOR YOURSELF, PUT THEM HERE:

\newcommand\E{\mathbb{E}}
\newcommand\dd{\mathrm{d}}

\usepackage{hyperref}
\usepackage{cancel}
\usepackage{listings}
\newcommand\pp{\partial}
\newcommand\pd{\partial}
\newcommand\imp{$\Longrightarrow$}
\newcommand\lb{\left (}
\newcommand\rb{\right )}
\newcommand\lsb{\left [}
\newcommand\rsb{\right ]}
\newcommand\lcb{\left \{}
\newcommand\rcb{\right \}}

\begin{document}

%\lecture{**LECTURE-NUMBER**}{**DATE**}{**LECTURER**}{**SCRIBE**}
\lecture{7}{Review Session \#7}{Aliaksandr Zaretski}{Zhikun Lu}
\footnotetext[1]{Visit \url{http://www.luzk.net/misc} for updates.}

\hfill Date: Date: September 6, 2018

%\tableofcontents
\section{Systems of Linear ODE with Constant coeffiicients}
\begin{theorem}
    \begin{equation}
    \begin{cases}
        \dot{x}(t) = A(t)x(t)+ B(t)u(t) \qquad [C]\\
        \dot{x}(t) = A(t)x(t)  \qquad [H]
    \end{cases}
    \end{equation}
    Then $e^{At}$ is a fundamental matrix of [H]. $e^{A(t-t_0)}$ is a state transtition matrix.
\end{theorem}
\begin{proof}
    $e^{At}$ is invertible, $(e^{At})^{-1} = e^{-At}$. And $\frac{\dd}{\dd t}(e^{At}) = A e^{At}$. By Theorem 6.9, we know $e^{A(t-t_0)} = e^{At} (e^{At_{0}})^{-1}$ is a state transtition matrix.
\end{proof}

\begin{theorem}
    General solution to [H] is $x(t) = e^{A(t-t_{0})}x(t_{0})  $.

    General solution to [C] is $x(t) = e^{A(t-t_{0})}x(t_{0}) + \int_{t_{0}}^{t} e^{A(t-s)}B(s)u(s) \dd s$.
\end{theorem}

\textbf{Exercise}: Find the general solution of \[
    \begin{bmatrix}
        \dot{x}_{1}(t)\\
        \dot{x}_{2}(t)
    \end{bmatrix}
    =
    \begin{bmatrix}
        a_{1} & 0\\
        0 & a_{2}
    \end{bmatrix}
    \begin{bmatrix}
        {x}_{1}(t)\\
        {x}_{2}(t)
    \end{bmatrix}
    +
    \begin{bmatrix}
        1 & 0\\
        0 & 1
    \end{bmatrix}
    \begin{bmatrix}
        u_{1}(t)\\
        u_{2}(t)
    \end{bmatrix}
\]

\section{Stability of Systems of ODE}
\begin{theorem}
    Consider a linear system $\dot{x}(t) = A x (t)$
    \begin{enumerate}
        \item [(a)] 0 is a unique equilibrium point $\iff$ $\det A \neq 0$
        \item [(b)] 0 is a stable equilibrium point $\iff$ all eigenvalues of $A$ have negative real parts.
    \end{enumerate}
\end{theorem}
\begin{proof}
    (a) Suppose $Ax = 0$ 

    ``\imp'' part: $Ax \neq 0$ for any $x \neq 0$ \imp $\det(A) \neq 0$. Contradiction!

    ``$\Longleftarrow$'' part: $Ax = 0$ \imp $x=0$ is the unique equilibrium point.

    (b) Suppose $A$ is diagonalizable [if not, a similar argument can be developed using Jordan decomposition]. So $A = P D P^{-1}$ \imp $\dot{x} = P D P^{-1} x $ \imp $\underbrace{P^{-1} \dot{x}}_{y} = D \underbrace{ P^{-1} x}_{\dot{y}}$ $\iff$ $\dot{y} = D y$.

    Note that $\lim_{t \to \infty} x(t) = 0 \iff \lim_{t \to \infty} y(t) = 0 $.

    Now look at $\dot{y} = D y$:
    $$\dot{y} = \begin{bmatrix}
        \dot{y}_{1}\\
        \vdots\\
        \dot{y}_{n}
    \end{bmatrix} = 
    \begin{bmatrix}
        d_{1} & &0 \\
            & \ddots &\\
        0&& d_{n}
    \end{bmatrix}
    \begin{bmatrix}
        y_{1}\\
        \vdots\\
        y_{n}
    \end{bmatrix}
    \iff 
    \dot{y}_{j} = d_{j} y_{j}, ~ j = 1, ..., n.$$
    $y_{j}(t) = c_{j} e^{\alpha_{t} t} = c_{j} e^{(a_{t}+ i b_{j}) t}.$
    Then $\lim_t y_{j}(t) = 0 \iff a_{j} < 0 
    $. Hence the conclusion follows.
\end{proof}

\begin{theorem}
    Consider a non-linear system $\dot{x} = f(x), ~ f : \mathbb{Q}^{n} \to \mathbb{R}^{n}$. Suppose $f$ is $C^{1}$ and $\tilde{x}$ is an equilibrium point. Then
    \begin{itemize}
        \item [(a)] all eigenvalues of $Df(\tilde{x})$ havenegative real parts \imp $\tilde{x}$ is locally asymptotically stable.
        \item [(b)] at least one eigenvalue of  $Df(\tilde{x})$ has positive real part \imp $\tilde{x}$ is unstable.
    \end{itemize}
\end{theorem}
\begin{proof}
    (For a rigorous argument, see Hartman-Grobman theorem.)

    By Taylor theorem,
    \[
    \underbrace{f(x)}_{\dot{x}} = \underbrace{f(\tilde{x})}_{=0}+ Df(\tilde{x})(x-\tilde{x}) + \Gamma (x - \tilde{x})
    \]
    and $\lim_{x\to \tilde{x}} \frac{\Gamma (x - \tilde{x})}{||x - \tilde{x}||} = 0$.
    \[
    \Longrightarrow \dot{x} = D f (\tilde{x}) (x- \tilde{x}) + \Gamma (x-\tilde{x})
    \]
    \[
    \iff \frac{\dd}{\dd t}(x - \tilde{x}) = Df(\tilde{x})\underbrace{(x-\tilde{x})}_{y} + \Gamma (x- \tilde{x})
    \]
    \[
    \Longrightarrow \dot{y} = D f(\tilde{x}) y + \Gamma (y)
    \]
    Then when $x$ is close to $\tilde{x}$, or $y $ is close to $0$, the stability of the original system is determined by the stability of its linearized version: $\dot{y} = Df(\tilde{x})y$. The conclusion follows from Theorem 7.3.
\end{proof}

\subsection*{Phase Diagrams}
\begin{equation}
    \begin{cases}
        \dot{x_{1}} = f_{1}(x_{1}, x_{2})\\
        \dot{x_{2}} = f_{2}(x_{1}, x_{2})
    \end{cases}
\end{equation}
\begin{example}
    \begin{equation}
    \begin{cases}
        \dot{x_{1}} = x_{2}\\
        \dot{x_{2}} = x_{1}
    \end{cases} \iff
    \begin{bmatrix}
        \dot{x_{1}} \\
        \dot{x_{2}} 
    \end{bmatrix} = 
    \begin{bmatrix}
        0&1\\
        1&0
    \end{bmatrix}
    \begin{bmatrix}
        \dot{x_{1}} \\
        \dot{x_{2}} 
    \end{bmatrix}
\end{equation}
\begin{center}
    \texttt{[Insert a phase diagram here]}
\end{center}
\end{example}
\begin{remark}
    Saddle path correspondences to an eigenvector with eigenvalue that has negative real part.
\end{remark}

\begin{center}
    \texttt{[The rest part is missing]}    
\end{center}

% Assume $x$ is a solution to [H] and $\Phi$ is invertible at $t_0$. Then we can show $ \tilde{x}(t) = \Phi(t) \Phi^{-1}(t_0) x(t_0) $ is also a solution to [H]. $ \tilde{x} $ and $x$ has the same initial condition because $ \tilde{x}(t_0) = \Phi(t_0) \Phi^{-1}(t_0) x(t_0) = x(t_0)$ . Hence by uniqueness, $ x(t) = \tilde{x}(t)$. Hence
% $$ x(t) = \tilde{x}(t) = \Phi(t) \Phi^{-1}(t_0) x(t_0) = \Phi(t) c .$$

%=======================================































%=======================================

%$$##
%\clearpage
%\section*{References}
%\beginrefs
%\bibentry{CW87}{\sc D.~Coppersmith} and {\sc S.~Winograd}, 
%``Matrix multiplication via arithmetic progressions,''
%{\it Proceedings of the 19th ACM Symposium on Theory of %Computing},
%1987, pp.~1--6.
%\endrefs

% **** THIS ENDS THE EXAMPLES. DON'T DELETE THE FOLLOWING LINE:

\end{document}
