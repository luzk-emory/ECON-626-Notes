% See instructions in the preamble file
\documentclass[twoside]{article}
%
% This is a borrowed LaTeX template file for lecture notes for CS267,
% Applications of Parallel Computing, UCBerkeley EECS Department.
% Now being used for CMU's 10725 Fall 2012 Optimization course
% taught by Geoff Gordon and Ryan Tibshirani.  When preparing 
% LaTeX notes for this class, please use this template.
%
% To familiarize yourself with this template, the body contains
% some examples of its use.  Look them over.  Then you can
% run LaTeX on this file.  After you have LaTeXed this file then
% you can look over the result either by printing it out with
% dvips or using xdvi. "pdflatex template.tex" should also work.
%

\setlength{\oddsidemargin}{0.25 in}
\setlength{\evensidemargin}{-0.25 in}
\setlength{\topmargin}{-0.6 in}
\setlength{\textwidth}{6.5 in}
\setlength{\textheight}{8.5 in}
\setlength{\headsep}{0.75 in}
\setlength{\parindent}{0 in}
\setlength{\parskip}{0.1 in}

%
% ADD PACKAGES here:
%

\usepackage{amsmath,amsfonts,graphicx}

%
% The following commands set up the lecnum (lecture number)
% counter and make various numbering schemes work relative
% to the lecture number.
%
\newcounter{lecnum}
\renewcommand{\thepage}{\thelecnum-\arabic{page}}
\renewcommand{\thesection}{\thelecnum.\arabic{section}}
\renewcommand{\theequation}{\thelecnum.\arabic{equation}}
\renewcommand{\thefigure}{\thelecnum.\arabic{figure}}
\renewcommand{\thetable}{\thelecnum.\arabic{table}}

%
% The following macro is used to generate the header.
%
\newcommand{\lecture}[4]{
   \pagestyle{myheadings}
   \thispagestyle{plain}
   \newpage
   \setcounter{lecnum}{#1}
   \setcounter{page}{1}
   \noindent
   \begin{center}
   \framebox{
      \vbox{\vspace{2mm}
    \hbox to 6.28in { {\bf Econ 626: Quantitative Methods II
  \hfill Fall 2018} }
       \vspace{4mm}
       \hbox to 6.28in { {\Large \hfill Lecture #1: #2  \hfill} }
       \vspace{2mm}
       \hbox to 6.28in { {\it Lecturer: #3 \hfill Scribes: #4} }
      \vspace{2mm}}
   }
   \end{center}
   \markboth{Lecture #1: #2}{Lecture #1: #2}

   %{\bf Note}: {\it LaTeX template courtesy of UC Berkeley EECS dept.}

   {\bf Disclaimer}: {\it Zhikun is fully responsible for the errors and typos appeared in the notes.}
   \vspace*{4mm}
}
%
% Convention for citations is authors' initials followed by the year.
% For example, to cite a paper by Leighton and Maggs you would type
% \cite{LM89}, and to cite a paper by Strassen you would type \cite{S69}.
% (To avoid bibliography problems, for now we redefine the \cite command.)
% Also commands that create a suitable format for the reference list.
\renewcommand{\cite}[1]{[#1]}
\def\beginrefs{\begin{list}%
        {[\arabic{equation}]}{\usecounter{equation}
         \setlength{\leftmargin}{2.0truecm}\setlength{\labelsep}{0.4truecm}%
         \setlength{\labelwidth}{1.6truecm}}}
\def\endrefs{\end{list}}
\def\bibentry#1{\item[\hbox{[#1]}]}

%Use this command for a figure; it puts a figure in wherever you want it.
%usage: \fig{NUMBER}{SPACE-IN-INCHES}{CAPTION}
\newcommand{\fig}[3]{
      \vspace{#2}
      \begin{center}
      Figure \thelecnum.#1:~#3
      \end{center}
  }
% Use these for theorems, lemmas, proofs, etc.
\newtheorem{theorem}{Theorem}[lecnum]
\newtheorem{lemma}[theorem]{Lemma}
\newtheorem{proposition}[theorem]{Proposition}
\newtheorem{claim}[theorem]{Claim}
\newtheorem{corollary}[theorem]{Corollary}
\newtheorem{definition}[theorem]{Definition}
%\newtheorem{example}[theorem]{Example}
\newenvironment{proof}{{\bf Proof:}}{\hfill\rule{2mm}{2mm}}
\newenvironment{example}{{\bf Example:}}{\hfill\rule{2mm}{2mm}}

\newtheorem{remark}[theorem]{Remark}
%\newenvironment{remark}[1][Remark]{\begin{trivlist}\item[\hskip \labelsep {\bfseries #1}]}{\end{trivlist}}


% **** IF YOU WANT TO DEFINE ADDITIONAL MACROS FOR YOURSELF, PUT THEM HERE:

\newcommand\E{\mathbb{E}}
\newcommand\dd{\mathrm{d}}


\begin{document}
%FILL IN THE RIGHT INFO.
%\lecture{**LECTURE-NUMBER**}{**DATE**}{**LECTURER**}{**SCRIBE**}
\lecture{1}{Review Session \#1}{Aliaksandr Zaretski}{Zhikun Lu}
%\footnotetext{These notes are partially based on those of Nigel Mansell.}

\begin{center}
  {\huge {\bf Differential Equations}}  
\end{center}
\section{Basic concepts}
\begin{definition}[ODE]
    Let $E \in \mathbb{R}$ and $ x \in E \rightarrow \mathbb{R}$ be an unknow function. An ODE of order n is an equation of the from
    $$F(t,x,x',x'', ... , x^{(n)}) = 0$$
    where $F(\cdot)$ is known, real-valued.
\end{definition}

This is an implicit equation. We will work with the explicit equations:
\[
x^{(n)} = f(t,x,x',x'', ... , x^{(n-1)})
\]
$f(\cdot)$ is an known real-valued function.

\begin{definition}[PDE]
    Let $E \in \mathbb{R}^k$. An PDE of order n is an equation of the from
    $$F(t,x, \mathrm{D} x,\mathrm{D}^2 x, ... , \mathrm{D}^{n}x) = 0$$
    where $F(\cdot)$ is known, real-valued.
\end{definition}

\begin{example} (ODE)
    \[
    [x'''(t)]^4 + e^{- \xi t} x''(t) + x(t) = \tan t
    \]
    Here, the order is 3, the degree is 4., exogenous variable: $t$, endogenous variable: $x$.
\end{example}

\begin{theorem}
    Let $E \in \mathbb{R}^2$, and $f: E \rightarrow \mathbb{R}$. If $f$ is continuously differentiable ($C^1$) at $(t_0, x_0) \in E$, then $\exists \epsilon > 0$ and a unique $C^1$ function $t \mapsto x(t)$, such that 
    $$x'(t) = f(t,x(t)), \forall t \in (t_0- \epsilon, t_0 + \epsilon),$$ and also $x(t_0) = x_0$
\end{theorem}

\begin{remark} 
    A \underline{general solution} to an ODE is a set of all solutions.
    A \underline{particular solutions} is a solution that satisfies \underline{initial conditions}.
\end{remark}

The number of initial conditions must be equal to the order of an ODE. For example, if you solve 
$$
x^{(n)}(t) = f(t, x', ..., x^{(n-1)})
$$
provide $x(t_0),x'(t_0), ..., x^{(n-1)}(t_0).$

\begin{definition}
    An ODE is linear if it takes the form $\mathrm{L}x(t) = g(t)$, where g is a known real-valued function, L is the linear differential operator,
    \[
    \mathrm{L} = a_0(t) + a_1(t)\frac{\dd }{\dd t}+ ... + a_n(t)\frac{\dd^n }{\dd t^n}.
    \]
    Then $$\mathrm{L}x(t) = a_0(t) x(t) + a_1(t)\frac{\dd x(t)}{\dd t}+ ... + a_n(t)\frac{\dd^n x(t)}{\dd t^n} = g(t)$$
\end{definition}

\begin{definition}
    An ODE is nonlinear if it is not linear.
\end{definition}

\section{Some common types of ODE}
\subsection{Separable ODEs}
\begin{equation}
    \begin{aligned}
        x'(t) = f(x)g(t) &\Longrightarrow \frac{\dd x}{\dd t} = f(x)g(t) \Longrightarrow
        \frac{\dd x}{f(x)}  = g(t) \dd t\\
        &\Longrightarrow \int \frac{\dd x}{f(x)}  = \int g(t) \dd t
    \end{aligned}
\end{equation}

\begin{example}
    \begin{equation}
    \begin{aligned}
        \frac{\dd x}{\dd t} = \frac{x}{t} \iff \int \frac{\dd x}{x} = \int \frac{ \dd t}{t}
        \Longrightarrow 
        \ln |x| = \ln |t| + C_1
        &\Longrightarrow 
        |x| = |t| e^{C_1} \equiv C_2 |t|\\
        &\Longrightarrow x = C_3 t
    \end{aligned}
    \end{equation}
    Suppose $x(1) = 5$, then $|x| = C_2 |t| \Longrightarrow x(t) = 5t$.
\end{example}

\subsection{Reducible to separable} 
Suppose we have
\[
x'(t) = f(a x + b t + c)
\]
Let $z = a x + b t + c$, then $z'(t) = a x'(t)+b = f(z) + b$, then $\frac{\dd z}{a f(z) + b} = \dd t$,
\[
    \int \frac{\dd z}{a f(z) + b} = \int \dd t = t + C_1
    \Longrightarrow \text{solve for } z(t) \Longrightarrow \text{solve for } x(t)
\]

\begin{example}
    Let $a = -1, b=1, c= 0$,
    \[
    \frac{\dd x}{\dd t} = \frac{1}{t-x} + 1
    \]
    The left hand side seems not separable. Let $z = t-x$, so $f(z) = \frac{1}{z}+1$. Then 
    \[
    \int \frac{\dd z}{-\frac{1}{z}-1+1} = t + C_1
    \iff \int z \dd z = -t - C_1 \Longrightarrow \frac{z^2}{2} = -t + C_2 \iff {z^2} = -2t + C_3
    \]
    \begin{equation}
    x(t) = t \pm \sqrt{-2t + C_3} \quad \text{- general solution}
    \end{equation}
    With initial condition $x(0) = 5$, then $5 = \pm \sqrt{C_3} \Longrightarrow C_3 = 25$. Hence
    \begin{equation}
        x(t) = t + \sqrt{-2t + 25} \quad \text{- particular solution}
    \end{equation}
\end{example}

\subsection{Homogeneous ODEs}
Homogeneous ODEs have the following form
\[
\frac{\dd x}{\dd t} = f(\frac{x}{t})
\]
Let $z = \dfrac{x}{t} \Longrightarrow x(t) = t z(t), x'(t) = z(t) + t z'(t) $. 
Hence, the original ODE can be transformed into
\begin{equation}
    z(t) + t z'(t) = f(z) \Longrightarrow \frac{\dd z}{f(z) - z} = \frac{\dd t}{t}
\end{equation}
\begin{equation}
    \int \frac{\dd z}{f(z) - z} = \ln |t| + C
\end{equation}

\begin{example}
    $\frac{\dd x}{\dd t} = \tan(\frac{x}{t}) + \frac{x}{t}$. So $ f(z) = \tan z + z $. Hence
    \begin{equation}
        \int \frac{\dd z}{\tan z} = \ln |t| + C \Longrightarrow \ln | \sin z| = \ln |t| + C \iff |\sin z | = C_1 |t| \iff z(t) = \arcsin (C_2 t)
    \end{equation}
    \begin{equation}
        \Longrightarrow \frac{x(t)}{t} = \arcsin (C_2 t) \Longrightarrow x(t) = t \arcsin (C_2 t) \quad \text{-\quad general solution}
    \end{equation}
    Now suppose $x(t_0) = x_0$, ..., 
    \begin{equation}
        \Longrightarrow x(t) = t \arcsin (\frac{t}{t_0}\sin(\frac{x_0}{t_0}))
    \end{equation}
\end{example}



































%$$##
\clearpage

\section*{References}
%\beginrefs
%\bibentry{CW87}{\sc D.~Coppersmith} and {\sc S.~Winograd}, 
%``Matrix multiplication via arithmetic progressions,''
%{\it Proceedings of the 19th ACM Symposium on Theory of %Computing},
%1987, pp.~1--6.
%\endrefs

% **** THIS ENDS THE EXAMPLES. DON'T DELETE THE FOLLOWING LINE:

\end{document}



